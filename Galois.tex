% !TeX encoding = UTF-8
\documentclass[Unicode]{ctexbook}
\usepackage{mathrsfs,amsfonts,amssymb,xcolor,amsmath,mathtools,bm,tikz,booktabs,tikz-cd,latexsym}
\tikzcdset{
	arrow style=tikz,
	diagrams={>={Straight Barb[scale=1.5]}}
}
\usepackage{fontspec}
\setmainfont{CMU Serif} %多语言支持
\usepackage{pdfpages} %插入PDF
\usetikzlibrary{angles} %画直角
\usepackage{tipa} %输入音标
\usepackage{metalogo} %输入\XeLaTeX
\usepackage{tabularx}
\usepackage[numbib]{tocbibind}
\usepackage{multicol}
\usepackage{makeidx}
\usepackage{imakeidx}
\makeindex[intoc]
\usepackage[hyperindex]{hyperref}
\hypersetup{
	colorlinks=true,
	linkcolor=purple,
	filecolor=cyan,      
	urlcolor=cyan,
	citecolor=red,
	bookmarks=true,
}
\usepackage{cite}
\bibliographystyle{plain}
\usepackage{enumitem}
\setenumerate[1]{itemsep=0pt,partopsep=0pt,parsep=\parskip,topsep=5pt,label=(\roman*)}
\setitemize[1]{itemsep=0pt,partopsep=0pt,parsep=\parskip,topsep=5pt}
\setdescription{itemsep=0pt,partopsep=0pt,parsep=\parskip,topsep=5pt}
%\usepackage{graphicx}
%\usepackage{wrapfig}
\title{\textbf{Galois理论}}
\author{\texttt{Ihaku}}
\date{}
\usepackage[
	type={CC},
	modifier={by-nc-sa},
	lang={chinese-utf8},
	version={4.0},
]{doclicense}
%\makeatletter %使\section中的内容左对齐
%\renewcommand{\section}{\@startsection{section}{1}{0mm}
	%{0.5\baselineskip}{0.5\baselineskip}{\bf\leftline}}
%\makeatother
\renewcommand{\baselinestretch}{1.5}
\usepackage[a4paper,left=1.25in,right=1.25in,top=1in,bottom=1in]{geometry}  
%\pagestyle{plain}
\renewcommand{\thefootnote}{{\color{red}\roman{footnote}}}
\tikzset{elegant/.style={smooth,samples=50,cyan}}
\usepackage[hyperref]{ntheorem}

\theorembodyfont{}
\newtheorem{theorem}{定理}[section] % 按section编号
\newtheorem{corollary}[theorem]{推论}
\theorembodyfont{}
\newtheorem{lemma}[theorem]{引理}
\theorembodyfont{}
\newtheorem{proposition}[theorem]{命题}
\theorembodyfont{}
\newtheorem{definition}[theorem]{定义}
\newtheorem{definition-theorem}[theorem]{定义--定理}
\newtheorem{definition-proposition}[theorem]{定义--命题}
\newtheorem{hypothesis}[theorem]{假设}
\newtheorem{conjecture}[theorem]{猜想}
\theorembodyfont{}
\newtheorem{example}[theorem]{例}
\newtheorem{remark}[theorem]{注记}
\newtheorem{convention}[theorem]{约定}
\newtheorem{exercise}[theorem]{练习}		% 穿插于文中的习题
\theorembodyfont{}
\theoremstyle{nonumberplain}
\theoremseparator{\enspace}
\theoremsymbol{\openbox} % 模拟标准的 Proof 环境
\newtheorem{proof}{证明}
\DeclareMathOperator{\Hom}{\mathrm{Hom}}
\DeclareMathOperator{\Aut}{\mathrm{Aut}}
\DeclareMathOperator{\End}{\mathrm{End}}
\DeclareMathOperator{\Inv}{\mathrm{Inv}}
\DeclareMathOperator{\identity}{\mathrm{id}}
\DeclareMathOperator{\Image}{\mathrm{Im}}
\DeclareMathOperator{\Ker}{\mathrm{Ker}}
\DeclareMathOperator{\ord}{\mathrm{ord}}
\DeclareMathOperator{\sgn}{\mathrm{sgn}}
\DeclareMathOperator{\Gal}{\mathrm{Gal}}
\begin{document}
	\frontmatter
	\maketitle
	\thispagestyle{empty}
	\vspace*{\fill}
	\doclicenseThis
	\chapter{前言}
本讲义是对于伽罗瓦理论的一个简要概括.

伽罗瓦理论是由法国数学家埃瓦里斯特·伽罗瓦(Évariste Galois)在19世纪创立的理论.其所有内容都包含在\cite{Galois}中,而该论文的艰涩难懂导致其险些被历史所湮没,伽罗瓦在此论文中使用置换群来描述给定的多项式的根与系数间的关系.其之后由戴德金(Julius Wilhelm Richard Dedekind),利奥波德·克罗内克(Leopold Kronecker),埃米尔·阿廷(Emil Artin)在此论文的基础上用现代语言改进并完善了伽罗瓦理论,形成了我们现在看到的形式,即伽罗瓦对应\ref{thm:finite-Galois-corr}.所以本讲义所介绍的即为现代伽罗瓦理论,而删繁就简,省去了许多历史渊源.对于此,可参考\cite{Artin}\cite{YH}.

本讲义例子较为稀少,少部分记号和定义未给出详细说明,多数定理未给出(严格)证明,其中人名定理都已标注,请自行参考文献和互联网资源.

除参考文献外,本讲义主要参考维基百科,采用的记号和翻译也以主流为准,同时在有歧义时给出注解.多数记号命名思路可参考索引中的中英对照,同时记号尽简洁而避繁杂,便于记忆,采取全文通用的形式,非首次出现则不再解释.

同时本讲义具备了目录,索引,参考文献,网址超链接,可谓麻雀虽小五脏俱全.对于文中的交叉引用也附有超链接,但考虑到线性阅读的流畅性,尽量减少交叉引用的使用,而是通过调整结构顺序来避免这一麻烦.

附录A为语音学相关知识,和伽罗瓦理论无关,仅供感兴趣的读者阅读.由于本人对法语一窍不通,如有纰漏,敬请指正.

考虑到\TeX 对多语言的支持较为困难(如希伯来语),且表格绘制过于繁琐,附录B是由PowerPoint制作,作为PDF文件添加到\TeX 中的.这里简单用到了\texttt{pdfpages}宏包,意外地好用,整体风格上也不显得过于突兀.

本讲义是对\LaTeX 编译的一次尝试,很多功能和宏包是第一次使用.感谢李文威老师提供的\cite{LW}之\TeX 源码(见 \url{https://gitee.com/wen-wei-li}),这对编写一份不算厚的讲义来说省去了大量的工夫.但也正因如此,存在着不少小问题.同时,李文威老师的写作功底也十分扎实,语言流畅自然,我非常推荐在写作前阅读李文威老师主页(见\url{https://wwli.asia})上的数学写作漫谈.

在编译本讲义的过程中我也学习到很多有趣的技巧,诸如超链接和索引是我认为编写现代阅读物不可或缺的功能,而这主要依赖于\texttt{hyperref}和\texttt{makeidx}宏包.这也是为何在体量不大的情形下仍旧使用目录和索引.

同时在编译过程中依旧遇到许多问题,有诸如宏包导入顺序出错而导致的功能残缺,和一些在其他文档中能正常编译却在本文档中出错的怪异现象.但是大部分问题都可通过多次尝试编译和网络和纸质文献的搜寻来解决,而对于尝试许久仍无法解决的问题,我想另辟蹊径不失为一个正确的选择.在讲义编写完成之际,我会将所有源码上传至GitHub,作为参考和备份.

GitHub页面: \url{https://github.com/Crystalxmz/Galois-Theory}

\begin{quotation}
	\textit{书山有路勤为径,学海无涯苦作舟.}
\end{quotation}

\rightline{\texttt{Ihaku}}

\rightline{\texttt{2022/2/27}}

	\tableofcontents
	\vspace*{\fill}
	\begin{center}
		\begin{tikzcd}[cells={nodes={draw=gray}},row sep=huge,column sep=huge]
		     	          & 0.1 \arrow[d] \arrow[rd] \arrow[ld] &     \\
			1.1 \arrow[rrddd,gray,dashed,bend right] & 0.2 \arrow[d] \arrow[ld]            & 1.2\arrow[d] \\
			1.4           & 2.1 \arrow[d] \arrow[rd] \arrow[ld] & 1.3\arrow[ddd,bend left] \\
			3.1	\arrow[d] \arrow[rd] & 2.2 \arrow[d]\arrow[dr] \arrow[r]                    & 2.3 \arrow[d]\\
			3.2			  & 3.3           &  4.1 \arrow[d]\arrow[ld]\\
			\text{A}.1 \arrow[d]&4.2 \arrow[r]&4.3\\
			%3.2&&\\
			\text{A}.2 \arrow[r]  & \text{A}.3\arrow[r] &\text{B} \\
		\end{tikzcd}
	\end{center}
	\vspace*{\fill}
	\mainmatter
	\setcounter{chapter}{-1}
\chapter{群环域概略}
本章是抽象代数的基本知识纲要,可看做群环域内容的最小闭包,仅可作为手册查阅.如需深入学习需翻阅抽象代数教材,辅以大量例子和习题.本章大量参考\cite{FK}\cite{LW}.
\section{群}
\begin{definition}\label{def:monoid}\index{banqun@半群 (semigroup)}
	集合$S$和$S$上满足结合律的二元运算 $\cdot$ 所形成的代数结构叫做\textbf{半群}.这个半群记成$(S,\cdot)$或者简记成$S$,运算$x\cdot y$也尝尝简写成$xy$.与任何元素相乘等于自身的称为\textbf{幺元},若含有幺元则称为\textbf{幺半群}\index{banqun@半群 (semigroup)!幺半群 (monoid)},幺元通常记作$e$或 $1$.若满足交换律则称为\textbf{交换半群}.
\end{definition}
\begin{example}
	$(\mathbb{Z},-)$不满足结合律,故不是半群.
\end{example}

对于交换幺半群,惯例是将其二元运算 $\cdot$ 写成加法 $+$,并将幺元$1$写成$0$,元素$x$的逆写成$-x$;但一些场合仍适用乘法记号.必要时另外申明.
\begin{definition}\label{def:group}\index{qun@群 (group)}
	与任何元素相乘等于幺元的称为\textbf{逆元},若含幺半群$(G,\cdot)$中每一个元素都存在逆元,则$G$叫做\textbf{群}.若满足交换律则称为\textbf{交换群}或\textbf{Abel群}\index{qun@群 (group)!Abel群 (Abel group)}.
\end{definition}

简言之,群内的元素满足封闭性,结合律,含幺元,含逆元四个性质.其中逆元往往难以满足,结合律通常难以验证.向量空间的前四条性质即是群的定义.
\begin{example}
	一个拓扑$(\tau,\cup)$是一个幺半群,而拓扑$(\tau,\triangle)$是一个群.单位元都为$\varnothing$,后者逆元为自身,亦即$\forall A\in\tau,A^2=\varnothing$,该群每一个非单位元的阶都为2.此为群中的拓扑,反之,拓扑中亦有群,称为\href{https://en.wikipedia.org/wiki/Topological_group}{\textbf{拓扑群}}.
\end{example}
\begin{definition}
	设 $G$ 为群, 子集 $H \subset G$ 被称为 $G$ 的\textbf{子群}\index{qun@群 (group)!子群 (subgroup)}, 如果
	\begin{enumerate}
		\item $H$ 是子幺半群,
		\item 对任意 $x \in H$ 有 $x^{-1} \in H$.
	\end{enumerate}
	表示成$H\leqslant G$.假若子群 $H$ 对所有 $x \in G$ 满足 $xH = Hx$, 则称 $H$ 为 $G$ 的\textbf{正规子群}\index{qun@群 (group)!子群 (subgroup)!正规子群 (normal subgroup)}, 记作 $H \lhd G$. 子群 $\{1\} \lhd G$ 称作 $G$ 的\textbf{平凡子群}.
\end{definition}
\begin{definition}\index{jie@阶 (order)}
	\begin{enumerate}
		\item 一个群的\textbf{阶}是指其势,即其元素的个数,记为$|G|$;
		\item 一个群内的一个元素$a$之\textbf{阶}(有时称为\textbf{周期})是指会使得$a^m = e$的最小正整数$m$.若没有此数存在,则称$a$有无限阶.有限群的所有元素有有限阶,记为$\ord a$.
	\end{enumerate}
\end{definition}
\begin{example}\label{eg:cyclic-group}\index{qun@群 (group)!循环群 (cyclic group)}
	包含$x$的最小群叫做由$x$\textbf{生成}的群,记作$\langle x\rangle$.若群 $G$ 中存在元素 $x$ 使得 $G = \langle x\rangle$, 则称 $G$ 为\textbf{循环群}. 循环群又叫单位生成群,且都同构于$\mathbb{Z}$的子群.
\end{example}
\begin{example}\label{eg:symmetric-group}\index{qun@群 (group)!对称群 (symmetric group)}\index{qun@群 (group)!置换群 (permutation group)}\index{qun@群 (group)!交错群 (alternating group)}
	从任意集合 $X$ 映到自身的全体双射构成一个群, 称为 $X$ 上的\textbf{对称群} $\mathfrak{S}_X \coloneqq \Aut(X)$. 其中的二元运算是双射的合成 $(f, g) \mapsto f \circ g$, 幺元为恒等映射 $\identity_X: X \to X$, 而逆元无非是逆映射. 当 $X = \{1, \ldots, n\}$ ($n \in \mathbb{Z}_{\geq 1}$) 时也称为 $n$ 次\textbf{对称群},记为 $\mathfrak{S}_n$\footnote{德文尖角体S,对应德文Symmetrische Gruppe或英文的首字母S.},它的每个子群称作\textbf{置换群}. 注意到 $|\mathfrak{S}_n| = n!$.其所有偶置换元素组成的子群称为\textbf{交错群},记作$\mathfrak{A}_n$\footnote{德文尖角体A,对应德文Alternierende Gruppe或英文的首字母A.},且$\mathfrak{A}_n\lhd \mathfrak{S}_n$.
\end{example}
\begin{definition}\label{def:coset}\index{peiji@陪集 (coset)}
	设 $H$ 为群 $G$ 的子群. 定义:
	\begin{enumerate}
		\item \textbf{左陪集}: $G$ 中形如 $xH$ 的子集, 全体左陪集构成的集合记作 $G/H$;
		\item \textbf{右陪集}: $G$ 中形如 $Hx$ 的子集, 全体右陪集构成的集合记作 $H \backslash G$;
		\item \textbf{双陪集}: 设 $K$ 为另一子群, 则 $G$ 中形如 $HxK \coloneqq \{hxk : h \in H, k \in K\}$ 的子集称为 $G$ 对 $(H,K)$ 的双陪集, 全体双陪集构成的集合记作 $H \backslash G/K$.
	\end{enumerate}
	陪集中的元素称为该陪集的一个代表元. $H \lhd G$ 等价于左,右陪集相同. 由于陪集的左右之分总能从符号辨明, 以下不再申明. 定义 $H$ 在 $G$ 中的\textbf{指数}
	\[ [G:H] \coloneqq |G/H|. \]
	陪集空间 $G/H$ 未必有限, 在此视 $[G:H]$ 为基数. 
\end{definition}
\begin{theorem}[Lagrange定理]\label{thm:Lagrange}
	设 $H$ 为 群 $G$ 的子群, 则
	\begin{enumerate}
		\item $|G| = [G:H] |H|$, 特别地, 当 $G$ 有限时 $|H|$ 必整除 $|G|$ ;
		\item 若 $K$ 是 $H$ 的子群, 则 $[G:K] =[G:H][H:K]$.
	\end{enumerate}
\end{theorem}
\begin{corollary}
	群$G$中任意元素$g$的阶整除$G$的阶,即$\mathrm{ord}\ g\mid|G|$.由此直接得费马小定理.
\end{corollary}

拉格朗日定理的逆命题并不成立.给定一个有限群$G$和一个整除$G$的阶的整数$d$,$G$并不一定有阶数为$d$的子群.最简单的例子是4次交替群$\mathfrak{A}_4$,它的阶是12,但对于12的因数6, $\mathfrak{A}_4$没有6阶的子群.对于这样的子群的存在性, Cauchy定理和Sylow定理给出了一个部分的回答.
\begin{definition}
	设 $G$ 为群.
	\begin{enumerate}
		\item $G$ 的\textbf{中心}定义为 $Z_G \coloneqq \{z \in G : \forall x \in G, \; xz=zx\}$
		\footnote{因其德文Zentrum(注意德文中名词首字母应大写),首字母为Z,也有部分书采用英文center的首字母C表示.};
		\item 设 $E \subset G$ 为任意子集, 定义其\textbf{中心化子}为 $Z_G(E) \coloneqq \{z \in G : \forall x \in E, \; xz=zx \}$
		\footnote{因其德文Zentralisator,首字母为Z,也有部分书采用英文centralizer的首字母C表示.};\index{zhongxinhuazi@中心化子 (centralizer)}
		\item 承上, 定义其\textbf{正规化子}为 $N_G(E) \coloneqq \{n \in G : nEn^{-1} = E \}$\footnote{因其德文Normalisator和英文normalizer,首字母为N.}.\index{zhengguihuazi@正规化子 (normalizer)}
	\end{enumerate}
	当 $E$ 是独点集 $\{x\}$ 时, 使用简写 $Z_G(x)$ 和 $N_G(x)$.
\end{definition}

显然有
\[
Z_G=Z_G(G)\leqslant Z_G(E)\leqslant N_G(E)\leqslant G.
\]

Abel群等价于中心是自身的群. $H\lhd G$等价于$N_G(H)=G$.
\begin{remark}\label{rem:HN}
	若 $N, H\leqslant G$, 而且 $H \subset N_G(N)$, 则 $HN = NH$ 是 $G$ 的子群且 $N \lhd HN$. 
\end{remark}
\begin{definition}\label{def:morphism}
设 $M_1, M_2$ 为幺半群. 映射 $\varphi: M_1 \to M_2$ 如满足下述性质即称为\textbf{同态}\index{tongtai@同态 (morphism)}
\begin{enumerate}
	\item $\forall x,y \in M_1, \; \varphi(xy) = \varphi(x) \varphi(y)$;
	\item $\varphi(1)=1$.
\end{enumerate}
从 $M_1$ 到 $M_2$ 的同态所成集合写作 $\Hom(M_1, M_2)$.设 $\varphi\in\Hom(M_1, M_2)$.它的\textbf{像}记作 $\Image(\varphi) \coloneqq \{\varphi(x) : x \in M_1\}$, 而其\textbf{核}定义为 \index{he@核 (kernel)}$\Ker(\varphi) \coloneqq \varphi^{-1}(1)$.若$M_1,M_2$是群,则他们分别是$M_1,M_2$的正规子群.

从幺半群 $M$ 映至自身的同态称为\textbf{自同态},自同态全体对加法和复合构成一个环,叫做\textbf{自同态环},记作$\End(M)\coloneqq\Hom(M,M)$.\index{tongtai@同态 (morphism)!自同态 (endomorphism)}同态的合成仍为同态. 取常值 $1$ 的同态称作\textbf{平凡同态}.

若存在同态 $\psi: M_2 \to M_1$ 使得 $\varphi \psi = \identity_{M_2}$, $\psi \varphi = \identity_{M_1}$, 则称 $\varphi$ 可逆而 $\psi$ 是 $\varphi$ 的逆; 可逆同态称作\textbf{同构}\index{tonggou@同构 (isomorphism)},记作$M_1\cong M_2$. 此时我们也称 $M_1$ 与 $M_2$ 同构. 从幺半群映至自身的同构称为\textbf{自同构},自同构全体构成一个群,叫做\textbf{自同构群},它是$\End(M)$的单位群(见\ref{def:unit}),记作$\Aut(M)\coloneqq U(\End(M))$,如恒等映射 $\identity_M\in\Aut(M)$.\index{tonggou@同构 (isomorphism)!自同构 (automorphism)}
\end{definition}
\begin{definition}\index{qun@群 (group)!商群 (quotient group)}
设 $G$ 为群, $N$ 为其正规子群. 在陪集空间 $G/N$ 上定义二元运算
\[ xN \cdot yN = xy N, \quad x,y \in G. \]
这使得 $G/N$ 构成一个群, 称为 $G$ 模 $N$ 的\textbf{商群}, 其中的幺元是 $1 \cdot N$ 而逆由 $(xN)^{-1} = x^{-1}N$ 给出. 群同态
\[
\pi: G\twoheadrightarrow G/N\footnote{一般用$\hookrightarrow$表示单射,用$\twoheadrightarrow$表示满射.可类比$\subset,\supset$记忆.},\qquad x \mapsto xN
\]
称为\textbf{商同态}.
\end{definition}
\begin{definition}\index{guidao@轨道 (orbit)}\index{wendinghuazi@稳定化子 (stabilizer)}\index{chuandi@传递 (transitive)}
	设幺半群 $M$ 作用于 $X$. 定义
	\begin{enumerate}
		\item \textbf{不动点}集 $X^M \coloneqq \{x \in X: \forall m \in M, \; mx=x \}$,有时也记作$\mathrm{Fix}_X(M)$;
		\item 对于 $x \in X$, \textbf{轨道} $Mx \coloneqq \{mx : m \in M \}$, 其元素称为该轨道的代表元, 轨道 $Mx$ 是 $X$ 的 $M$-子集; 若$M$在$X$上的作用只有一个轨道,则称其是\textbf{传递的}或\textbf{可迁的}.
		\item 承上, 其\textbf{稳定化子}定为 $M$ 的子幺半群 $M_x \coloneqq \{m \in M : mx=x\}$.
	\end{enumerate}
\end{definition}
\begin{theorem}[轨道分解定理]\label{prop:orbit-decomp}
	设群 $G$ 作用于 $X$, 则
	\begin{enumerate}
		\item 有轨道分解 $X = \bigsqcup_x Gx$, 其中我们对每个轨道选定代表元 $x$;
		\item 对每个 $x \in X$, 映射
		\[
			G/G_x \to Gx,\qquad g \cdot G_x \mapsto gx
		\]
		是 $G$-集间的同构;
		\item 特别地, 我们有基数的等式
		\[ |X| = \sum_x \left[ G: G_x \right]; \]
		\item 对所有 $x \in X$ 和 $g \in G$, 有
		\[ G_{gx} = g G_x g^{-1}. \]
	\end{enumerate}
\end{theorem}
\begin{definition}\label{def:conj-action} \index{gonge@共轭 (conjugation)}
	依旧设 $G$ 为群. 伴随自同构 $\mathrm{Ad}: G \to \Aut(G)$ 给出的作用称为 $G$ 的\textbf{共轭作用} $G \times G \to G$ (在此考虑左作用). 定义展开后无非是
	\[ (g, x) \longmapsto {}^g x \coloneqq gxg^{-1}. \]
	共轭作用下的轨道称为 $G$ 中的\textbf{共轭类}.
\end{definition}

推而广之, 对任意子集 $E \subset G$ 我们业已定义子群 $N_G(E)$, 它在 $E$ 上的作用也叫共轭. 若两子集 $E, E'$ 满足 $\exists g \in G, \; E' = gEg^{-1}$, 则称 $E$ 与 $E'$ 共轭.易知正规子群仅与自身共轭.

非交换群共轭作用的性状一般相当复杂.对于$x \in G$,其稳定化子群正是中心化子$Z_G(x)$,而不动点集则是中心 $Z_G$.剖析$G$的共轭作用是了解其群结构的必由之路.
\begin{theorem}[同态基本定理]\label{thm:1st-homomorphism}
	设 $\varphi\in\Hom(G, G')$,则 $\varphi$ 诱导出同构
	\[
	\bar{\varphi}: G/\Ker(\varphi) \to \Image(\varphi),\qquad g \cdot \Ker(\varphi)\mapsto\varphi(g).
	\]
	此同构叫做\textbf{正则同构}.
\end{theorem}
\begin{theorem}[Caylay定理]\label{thm:Caylay}
	对任意有限群$G$,同态
	\[
	\rho:G\to \mathfrak{S}_G,\qquad\rho(g)a=ga
	\]
	是单的,故$\Ker(\rho)=\{1\}$,利用同态基本定理得:每个群均同构于某个对称群的子群.
\end{theorem}
\begin{theorem}[Cauchy 定理]\label{thm:group-Cauchy}
	设 $G$ 为有限群, 素数 $p$ 整除 $|G|$, 则存在 $x \in G$ 使得 $\mathrm{ord}\ x=p$.
\end{theorem}
\begin{definition}\index{qun@群 (group)!子群 (subgroup)!Sylow $p$-子群 (Sylow $p$-subgroup)}
设 $G$ 为 $n$ 阶有限群, $p$ 为素数. 设 $p^m \mid n$, 满足 $|H| = p^m$ 的子群 $H$ 称为 $G$ 的 \textbf{Sylow $\bm{p}$-子群}.
\end{definition}
\begin{theorem}[Sylow 定理]对任意有限群 $G$ 和任意素数 $p$, 
	\begin{enumerate}
		\item $G$ 含有 Sylow $p$-子群.
		\item 
		\begin{enumerate}
			\item 任意 $p$-子群 $H \subset G$ 皆包含于某个 Sylow $p$-子群;
			\item $G$ 的任两个 Sylow $p$-子群 $P, P'$ 皆共轭;
		\end{enumerate}
		特别地, $G$ 中存在正规的 Sylow $p$-子群 当且仅当 $G$ 有唯一的 Sylow $p$-子群.
		\item $G$ 中 Sylow $p$-子群的个数 $\equiv 1 \pmod p$.
	\end{enumerate}
\end{theorem}
\begin{theorem}[有限生成Abel群结构定理]\label{thm:finitely generated Abel}
	有限生成Abel群都同构于若干$\mathbb{Z}$子群的直和.
\end{theorem}
\section{环和域}
\begin{definition}\index{huan@环 (ring)}
	称$(R, +, \cdot)$是\textbf{(含幺)环},如果
	\begin{enumerate}
		\item $(R, +)$ 是Abel群, 二元运算用加法符号记作 $(a,b) \mapsto a+b$, 加法幺元记为 $0$, 称之为 $R$ 的加法群;
		\item $(R, \cdot)$ 是含幺半群;
		\item $a(b+c) = ab+ac$, $(b+c)a = ba + ca$ \quad (分配律, 或曰双线性)
	\end{enumerate}

	除去和幺元相关性质得到的 $(R, +, \cdot)$ 称作无幺环.若子集 $S \subset R$ 对 $(+, \cdot)$ 也构成环, 并且和 $R$ 共用同样的乘法幺元 $1$, 则称 $S$ 为 $R$ 的\textbf{子环}\index{huan@环 (ring)!子环 (subring)}, 或称 $R$ 是 $S$ 的环扩张或扩环.若乘法也满足交换律则称为\textbf{交换环}.\index{huan@环 (ring)!交换环 (commutative ring)}
\end{definition}
\begin{example}
	一般将有限个元素 $r_1, \ldots, r_n \in R$ 生成的环记为 $\langle r_1, \ldots, r_n\rangle$. 在交换环的情形也习惯写作 $(r_1, \ldots, r_n)$.零环$(0)$是无幺环,也是平凡环.
\end{example}
\begin{definition}\index{tongtai@同态 (morphism)}
	设 $R, S$ 为环, 映射 $\varphi: R \to S$ 为\textbf{环同态},如果$\varphi$是加法群同态,且为乘法幺半群同态.如去掉与 $1_R$, $1_S$ 相关的条件, 就得到无幺环之间的同态概念.
\end{definition}

由此可导出环的同构 (即可逆同态), 自同态, 自同构,像与核等概念,与\ref{def:morphism}同一套路, 不再赘述.

\begin{definition}\label{def:ideals}\index{lixiang@理想 (ideal)}
	设 $R$ 为环, $I \subset R$ 为加法子群.
	\begin{enumerate}
		\item 若对每个 $r \in R$ 皆有 $rI \subset I$, 则称 $I$ 为 $R$ 的\textbf{左理想};
		\item 若对每个 $r \in R$ 皆有 $Ir \subset I$, 则称 $I$ 为 $R$ 的\textbf{右理想};
		\item 若 $I$ 兼为左, 右理想, 则称作\textbf{双边理想}.
	\end{enumerate}
	满足 $I \neq R$ 的左, 右或双边理想称为真理想. 交换环的左,右理想不分,与双边理想一起简称为理想.
\end{definition}
\begin{definition}
	设 $I$ 为 $R$ 的理想, 赋予加法群 $R/I$ 乘法运算如下
	\[ (r+I) \cdot (s+I) \coloneqq (rs + I), \quad r, s \in R. \]
	则 $R/I$ 构成一个环, 称为 $R$ 模 $I$ 的\textbf{商环}. 商映射 $R \twoheadrightarrow R/I$ 称为\textbf{商同态}.
\end{definition}
\begin{theorem}[环同态基本定理]\label{thm:1st-homomorphism-ring}
	设 $\varphi\in\Hom(R, R')$, 则 $\Ker(\varphi) \coloneqq \varphi^{-1}(0)$ 是 $R$ 的理想, 且诱导同态 $\bar{\varphi}: R/\Ker(\varphi) \to \Image(\varphi)$ 是环同构.
\end{theorem}
\begin{definition}\label{def:unit}
	既然 $R$ 对乘法构成幺半群, 故可定义其中元素的左逆与右逆. 设 $r \in R$ 非零, 若 $r$ 可逆, 其逆记为 $r^{-1}$; 全体可逆元构成的乘法群称作\textbf{单位}\index{danwei@单位 (unit)},记作$U(R)$,有时也简记 $R^\times$. 若存在 $r' \neq 0$ 使得 $rr'=0$ 则称 $r$ 为\textbf{左零因子}; 条件改作 $r'r=0$ 则称\textbf{右零因子}. 为 $R$ 中左或右零因子的元素统称为\textbf{零因子}. 元素 $r \in R-\{0\}$ 非左零因子当且仅当 $r$ 的左乘满足消去律; 右零因子的情形类似.\index{lingyinzi@零因子 (zero divisor)}
\end{definition}
\begin{definition}\label{def:ring-characteristic}\index{tezheng@特征 (characteristic)}
	设 $R$ 非零环, 定义其\textbf{特征}为加法群元素的最大阶,记为 $\text{char}(R)$.若含有无限阶元素则特征记为$0$.
\end{definition}
\begin{example}[Frobenius自同态]
	对于特征$p$域$K$,利用二项式定理和数论中的有关结论可知$\forall x,y\in K$,令$f(x)=x^p$,则
	\[
	f(x+y)=\left(x+y\right)^p=x^p+y^p=f(x)+f(y),
	\]
	故$f$是$K$上的自同态.
\end{example}
\begin{definition}\label{def:field}\index{huan@环 (ring)!整环 (integral domain)}\index{huan@环 (ring)!除环 (division ring)}\index{yu@域 (field)}
	无零因子的交换环称为\textbf{整环}.若环 $R$ 中的每个非零元皆可逆, 则称 $R$ 为\textbf{除环}. 交换除环称为\textbf{域}\footnote{1871年德国数学家戴德金提出Körper的概念,因此也有书中用$K$指代域而非$F$.在德语里该词意为``体'',故日本和港澳台地区从德文直译为汉字``体'',与大陆有所不同.1893年由美国数学家摩尔将Körper翻译为field,而大陆所采用的翻译由英语转译为域.}.可类似定义\textbf{子除环}和\textbf{子域}.
\end{definition}
\begin{example}
按本节的约定, 除环不能是零环,四元数$\mathbb{H}$是除环; 某些文献将除环称作\textbf{斜域}(skew field)或\textbf{体},但因体在不同地区有歧义,所以尽量避免使用体这一说法.
\end{example}
\begin{proposition}
	无零因子的有限环必为除环.
\end{proposition}
\begin{theorem}[Wedderburn小定理]
	对于有限环,整环等价于除环等价于域.
\end{theorem}
\begin{definition}\index{chengxingziji@乘性子集 (multiplicative subset)} \index{jubuhua@局部化 (localization)}
	设 $R$ 为交换环. 子集 $S \subset R$ 若对环的乘法构成幺半群, 则称 $S$ 为 $R$ 的\textbf{乘性子集}. 构作对乘性子集 $S$ 的\textbf{局部化} $R[S^{-1}]$ 如下. 首先在集合 $R \times S$ 上定义关系
	\[
	(r,s) \sim (r',s') \Leftrightarrow \left[ \exists t \in S, trs' = tr's \right]. 
	\]
	易证 $\sim$ 是等价关系, 相应的商集记为 $R[S^{-1}]$, 其中的等价类 $[r,s]$ 应该设想为``商'' $r/s$, 且对任意 $t \in S$ 皆有 $[r,s]=[rt,st]$. 以下定义的环运算因而是顺理成章的:
	\begin{gather*}
		[r,s] + [r',s'] = [rs' + r's, ss'], \\
		[r,s] \cdot [r',s'] = [rr', ss'].
	\end{gather*}
	$R[S^{-1}]$ 对此成交换环, 零元为 $0 = [0, s]$ 而幺元为 $1 = [s,s]$, 其中 $s \in S$ 可任取. 由此得到
	\begin{gather*}\label{eqn:localization-zero}
		[r,s]=0 \Leftrightarrow \left[ \exists t \in S, tr=0 \right].
	\end{gather*}
	因此 $R[S^{-1}]$ 是零环当且仅当存在 $s \in S$ 使得 $sR=0$, 我们既假定 $R$ 含幺元, 这也相当于说 $0 \in S$; 一般总排除这种情形. 
	
	另一方面, $r \mapsto [r,1]$ 给出环同态 $R \to R[S^{-1}]$. 注意到 $s \in S$ 的像落在 $R[S^{-1}]^\times$ 中, 其逆无非是 $[1,s]$. 局部化应当同态射 $R \to R[S^{-1}]$ 一并考量.
\end{definition}
\begin{lemma}\label{prop:localization-units}
	设 $S \subset R$ 为乘性子集, $0 \notin S$, 则 $[r,s] \in R[S^{-1}]$ 可逆当且仅当存在 $r_1 \in R$ 使得 $rr_1 \in S$.
\end{lemma}
\begin{proof}
	若 $rr_1 \in S$ 则 $[r,s] [r_1 s, rr_1] = 1$. 反之设存在 $[r',s']$ 使得 $[r,s][r',s'] = 1$, 则存在 $t \in S$ 使得 $trr' = tss'$, 因而 $r(tr') \in S$. 
\end{proof}

原环 $R$ 的部分信息可能在局部化过程中丢失. 可知
\[ \Ker \left[ R \to R[S^{-1}] \right] = \left\{r \in R : \exists s \in R, \; sr=0 \right\}. \]
我们希望取尽可能大的 $S$ 使得 $R[S^{-1}]$ 是 $R$ 的扩环. 前述讨论自然引向以下结果.
\begin{lemma}
	设 $S \subset R$ 为乘性子集, $0 \notin S$. 则局部化态射 $R \to R[S^{-1}]$ 是单射当且仅当 $S$ 不含零因子. 另一方面, $R - \{0\}$ 中的所有非零因子构成 $R$ 的乘性子集, 相应的局部化记为
	\[ R \hookrightarrow \text{Frac}(R), \]
	而 $\text{Frac}(R)$ 称为 $R$ 的\textbf{全分式环}.
	
	当 $R$ 是整环时, $\text{Frac}(R)$ 无非是对 $S := R - \{0\}$ 的局部化; 此时由引理 \ref{prop:localization-units} 知 $\text{Frac}(R)$ 是域: 事实上 $r \neq 0$ 时 $[r,s]^{-1} = [s,r]$; 称此为 $R$ 的\textbf{分式域}或\textbf{商域}. \index{yu@域 (field)!分式域 (field of fractions)}
\end{lemma}

局部化是交换代数中的常见操作, 它把环里一些元素变得可逆, 是分式域概念的推广. 在代数几何的观点下, 局部化所得的环是原来的环的某些``局部'', 其谱自然地是原来环的谱的子集. 既然如此, 局部化的环通常会变得更简单. 我们也常常通过研究环的各个局部化来研究环本身.
\begin{definition}\label{def:ring-ideal}\index{lixiang@理想 (ideal)!素理想 (prime ideal)}\index{lixiang@理想 (ideal)!极大理想 (maximal ideal)}
	含幺交换环 $R$ 的真理想 $I$ 称为
	\begin{enumerate}
		\item \textbf{素理想}, 如果 $xy \in I$ 蕴涵 $x \in I$ 或 $y \in I$;\footnote{有些书对于一般环的素理想定义为:对于$R$的理想$I$,如果任意两个理想$A,B$满足$AB\subset I$,则$A\subset I$或者$B\subset I$.当环是含幺交换环时这两种定义是等价的.}
		\item \textbf{极大理想}, 如果 $I \neq R$ 且不存在严格包含 $I$ 的理想.
	\end{enumerate}
	分别记 $R$ 中素理想和极大理想所成的集合为 $\mathrm{Spec}\ R$ 与 $\mathrm{MaxSpec}\ R$, 称为 $R$ 的素谱和极大理想谱.
\end{definition}
\begin{proposition}\label{prop:prime-maximal-ideals}
	设 $I$ 为含幺交换环$R$ 的真理想, 则
	\begin{enumerate}
		\item $R/I$ 为整环当且仅当 $I$ 为素理想;
		\item $R/I$ 为域当且仅当 $I$ 为极大理想.
	\end{enumerate}
\end{proposition}
\begin{corollary}\label{prop:maximal-implies-prime}
	极大理想必为素理想.其逆一般不成立, 因为整环未必是域.
\end{corollary}
\begin{definition}\label{def:PID}\index{zhenghuan@整环 (integral domain)!主理想整环 (PID, principal ideal domain)}
	设 $I$ 为 $R$ 的理想, 若存在 $a \in R$ 使得 $I = \langle a\rangle = Ra$, 则称 $I$ 为\textbf{主理想}. 若整环 $R$ 的所有理想皆为主理想, 则称 $R$ 为\textbf{主理想整环}.
\end{definition}

利用\textbf{主理想整环上有限生成模的结构定理},我们可以直接推得\ref{thm:finitely generated Abel}和中国剩余定理.中国剩余定理是初等数论中的常见定理,该定理用环论的语言表述如下:
\begin{theorem}[中国剩余定理]\label{prop:CRT}\index{zhongguoshengyudingli@中国剩余定理 (CRT, Chinese Remainder Theorem)}
	设 $R$ 为环, $I_1, \ldots I_n$ 为一族理想. 假设对每个 $i \neq j$ 皆有 $I_i + I_j = R$, 则环同态
	\begin{align*}
		\varphi: R \to \prod_{i=1}^n R/I_i, \qquad r \mapsto \left( r \bmod I_i \right)_{i=1}^n
	\end{align*}
	诱导出环同构 $R \big/ (\bigcap_{i=1}^n I_i) \cong \prod_{i=1}^n R/I_i$.
\end{theorem}
\begin{definition}\label{def:UFD}\index{bukeyue@不可约元 (irreducible element)}\index{zhenghuan@整环 (integral domain)!唯一分解整环 (UFD, unique factorization domain)}
	整环 $R$ 中的非零元 $r$ 称为\textbf{不可约元}, 如果 $r \notin R^\times$ 而且在 $R$ 中 $d \mid r$ 蕴涵 $\langle d\rangle = \langle r\rangle$ 或 $d \in R^\times$. 不可约性仅取决于 $r$ 在 $\mathcal{P}$ 中的像. 令$\mathcal{P}\coloneqq(R - \{0\})/R^\times$, 以 $\mathring{x} \in \mathcal{P}$ 标记 $x \in R - \{0\}$ 的像如果 $\mathcal{P}$ 的每个元素 $\mathring{r}$ 都能写成
	\[ \mathring{r} = \prod_{i=1}^n \mathring{p}_i, \quad n \in \mathbb{Z}_{\geq 0} \]
	其中 $\mathring{p}_i \in \mathcal{P}$ 不可约, 而且 $\{\mathring{p}_1, \dots, \mathring{p}_n \}$ (计重数但不计顺序) 是唯一的, 则称 $R$ 为\textbf{唯一分解整环}; 称 $\mathring{p}_1, \dots, \mathring{p}_n$ (或其原像 $p_1, \dots, p_n \in R$) 是 $\mathring{r}$ (或其原像 $r \in R$) 的不可约因子. 约定 $n=0\iff{\mathring{r}}=1$ .
	如果整环 $R$ 中的非零元 $p$ 满足 $p \notin R^\times$ 而且 $p \mid ab \iff (p \mid a) \vee (p \mid b)$, 则称 $p$ 是\textbf{素元}. \index{suyuan@素元 (prime element)}
\end{definition}

有以下结论:
\begin{enumerate}
	\item $p$是素元$\iff \langle p\rangle$是素理想;
	\item 素元是不可约元,当环是UFD时反之也成立;
	\item 整环$R$是UFD当且仅当主理想满足升链条件且不可约元皆为素元,前者保证不可约分解存在,后者保证此分解唯一.
	\item UFD中任意两个元素$a,b$都具有最大公因子$(a,b)$和最小公倍元$[a,b]$.
\end{enumerate}
\begin{definition}\label{def:Euclidean-ring}\index{zhenghuan@整环 (integral domain)!欧几里得整环 (ED, Euclid Domain)}
	设 $R$ 为整环, 若存在良序集$L$和函数 $N: R - \{0\} \to L$, 使得对任意 $x \in R$, $d \in R - \{0\}$ 都存在 $q \in R$ 使 $r := x-qd$ 满足 
	\[ r=0  \quad \text{或者}\quad r \neq 0 \text{且} N(r) < N(d). \]
	满足此条件的 $R$ 称作\textbf{欧几里得整环}.
\end{definition}
\begin{proposition}
	ED是PID, PID是UFD.
\end{proposition}

判定一个环是否为PID并不容易. ED推广了$\mathbb{Z}$中的带余除法,从而使判断PID变为判断ED,但并非时时好用,即存在非ED的PID,也存在非PID的UFD.
\begin{theorem}[裴蜀定理]
	对于PID中的任意元素$a,b$,存在$x,y$,使得以下等式成立:
	\[
	ax+by=(a,b).
	\]
\end{theorem}

交换环理论含有丰富的内容,详细请阅读\cite{Atiyah}\cite{Matsumura}.
%多项式环的内容请参照\cite[2.5]{FK}或\cite[5.6, 5.7]{LW},对称多项式环的内容请参照\cite[附录2.2]{FK}或\cite[5.8]{LW},这对于Galois理论的学习至关重要.
	\chapter{要点知识}
本章是对前一章的一个补充,多数内容是初次学习群环域所非必要的内容,但对掌握Galois理论和深入学习抽象代数又不可或缺.
\section{对称群}
\begin{definition}\index{xunhuan@轮换 (cycle)}\index{duihuan@对换 (transposition)}
	设 $a_1, \ldots, a_m$ 是 $X$ 中相异的元素. 对称群 $\mathfrak{S}_X$(见\ref{eg:symmetric-group}) 中的 $m$-\textbf{轮换}$(a_1 \cdots a_m)$ 是下述映射 $\sigma: X \to X$
	\begin{gather*}
		\sigma(a_i) = a_{i+1}, \quad i \in \mathbb{Z}/m\mathbb{Z}, \\
		\sigma(x)=x, \quad x \notin \{a_1, \ldots, a_m\},
	\end{gather*}
	在此将下标 $\{1, \ldots, m\}$ 方便地视为 $\mathbb{Z}/m\mathbb{Z}$ 中元素, 即模 $m$ 的同余类. 称 $m$ 为该轮换的长度; $2$-轮换 $(a b)$ 又称\textbf{对换}. 我们称 $\mathfrak{S}_X$ 中两个轮换 $(a_1 \cdots a_m),(b_1 \cdots b_k)$ 不交, 如果 $\{a_1, \ldots, a_m\} \cap \{b_1, \ldots, b_k\} = \varnothing$. 
\end{definition}

由先前讨论可知不交的轮换对乘法相交换. 同样显然的是 $\ord\ (a_1 \cdots a_m) = m$.

\begin{proposition}[轮换分解]\label{prop:cyclic-decomposition}
	每个 $\sigma \in \mathfrak{S}_X$ 都能表成不交的轮换之积
	\[ \sigma = (a_1 a_2 \cdots) (b_1 b_2 \cdots) \cdots \]
	其中的轮换 $(a_1 \cdots),(b_1 \cdots)$ 在至多差一个顺序的意义下唯一. 由于 $1$-轮换是单位元, 乘积中可以省去.
\end{proposition}

这无非是 $X$ 在 $\sigma$ 生成的有限轮换群 $\langle\sigma\rangle$ 下的轨道分解 (引理 \ref{prop:orbit-decomp}), 每个轮换对应到一个轨道, 描述了 $\sigma$ 在该轨道上的作用.

我们称轮换分解中出现的轮换长度 $n_1, n_2, \ldots$ (包括长度为一的轮换) 为 $\sigma$ 的\textbf{轮换型}, 计重数不计顺序. 为了得到唯一性, 不妨排成 $n_1 \geq n_2 \geq \cdots$, 轮换型因之对应于整数 $n \coloneqq |X|$ 的\textbf{分拆}: $n = n_1 + n_2 + \cdots$. 上面对阶数的讨论蕴涵 $\sigma$ 的阶数等于 $n_1, n_2, \ldots$ 的最小公倍数. \index{fenchai@分拆 (partition)}
\begin{corollary}[对换分解]\label{prop:S_n-generation}
	每个 $\sigma \in \mathfrak{S}_n$ 都能表成若干对换的积,但不唯一.且群 $\mathfrak{S}_n$ 由对换 $(1i)$或$(i-1\ i)$生成, 这里 $1 < i \leqslant n$.
\end{corollary}

我们既可以将$m$-轮换拆分成$m-1$个对换之积,也可以直接通过排序算法(如冒泡排序)将其拆分,行列式中的逆序数可看为选择排序.由于对换分解不唯一,且两两不可交换,故不如轮换分解方便.

据此, 共轭作用(见\ref{def:conj-action})在对称群情形下有干净的陈述.
\begin{lemma}\label{prop:cycle-type}
	设 $\tau = (a_1 a_2 \cdots) (b_1 \cdots) \cdots$ 为上述的轮换分解, $\tau \in \mathfrak{S}_X$, 则
	\[ \sigma \tau \sigma^{-1} = (\sigma(a_1) \sigma(a_2) \cdots) (\sigma(b_1) \cdots) \cdots. \]
	作为推论, 元素 $\tau$ 的共轭类由其轮换型确定; $\mathfrak{S}_X$ 中的共轭类一一对应于轮换型 $n_1 \geq n_2 \geq \cdots$, 后者又一一对应于整数 $n = |X|$ 的分拆.
\end{lemma}

这无非是先给一个新序,置换后再回到旧序,等价于在新序下的置换.
\begin{lemma}
	存在唯一的群同态 $\sgn: \mathfrak{S}_n \to \{\pm 1\}$ 使得 $\sgn((a b)) = -1$.
\end{lemma}

若置换$\sigma\in\Ker(\sgn)$,则称为偶置换,否则为奇置换.虽然对换分解不唯一,但对换分解个数的奇偶性将始终保持一致(因为两个对换之积为一个3-轮换,不可能退化成一个对换),如何得到置换的奇偶性在交错代数(比如行列式)中将非常关键.

显然奇偶置换个数相同,为此我们可以构造一个映射,将每个偶置换乘上随意一个对换则为奇置换,容易验证这是一个双射.因此$|\mathfrak{A}_n| = n!/2$.
\begin{definition}\index{qun@群 (group)!单群 (simple group)}
	只有平凡正规子群的群称为\textbf{单群}.
\end{definition}
\begin{example}
	\begin{enumerate}
		\item 素数阶循环群是单群,而$p^n(n\geqslant2,p\text{为素数})$阶群有非平凡中心,故不是单群;
		
		\item $pq,p^2q(p,q\text{为素数})$阶群不是单群;
		
		\item $2m(m\text{为大于3奇数})$阶群不是单群.
	\end{enumerate}
\end{example}

\begin{theorem}[É. Galois]\label{prop:A_n-simple}
	当 $n \geq 5$ 时 $\mathfrak{A}_n$ 是单群.
\end{theorem}
\begin{proof}
	设 $H \lhd \mathfrak{A}_n$, $H \neq \{1\}$. 从以上性质可知找出一个 $3$-轮换 $\sigma \in H$ 即足. 兹断言取 $\sigma \in H - \{1\}$ 使得 $\left|\text{Fix}(\sigma)\right|$ 极大便是.
	
	如果 $\sigma$ 的轮换分解中只有对换, 那么分解中至少含两项如 $(i j) (k l)$, 其中 $\{i,j\} \cap \{k,l\} = \emptyset$. 由于 $n \geq 5$, 可取 $r \notin \{i,j,k, l\}$ 并定义
	\begin{equation}\label{eqn:tau-sigma-simple}
		\tau \coloneqq (k l r), \quad \sigma' \coloneqq [\tau, \sigma] = \tau\sigma\tau^{-1}\sigma^{-1} \in H \quad (\because H \lhd \mathfrak{A}_n).
	\end{equation}
	可直接验证 $i,j \in \text{Fix}(\sigma') - \text{Fix}(\sigma)$, $\sigma'(k) = r \neq k$, 以及
	\[ \text{Fix}(\sigma) - \{r\} = \text{Fix}(\sigma) - \{k, l, r\} = \text{Fix}(\sigma) \cap \text{Fix}(\tau) \subset \text{Fix}(\sigma'). \]
	综之 $\left|\text{Fix}(\sigma')\right| > \left|\mathrm{Fix}(\sigma)\right|$, 矛盾.
	
	设 $\sigma$ 的轮换分解中包含长度 $>2$ 的项 $(i j k \cdots)$. 假若 $\sigma = (i j k)$ 则是所求的 $3$-轮换; 否则因为 $\sigma$ 不可能是 $4$-轮换, $\sigma$ 除了 $i,j,k$ 之外还挪动至少两个相异元 $r, l$. 依然以 \eqref{eqn:tau-sigma-simple} 式定义 $\sigma' \in H$. 可以验证 $j \in \text{Fix}(\sigma')$, $\sigma'(k) = l \neq k$ 和
	\[ \text{Fix}(\sigma) = \text{Fix}(\sigma) - \{k,l,r\} = \text{Fix}(\sigma) \cap \text{Fix}(\tau) \subset \text{Fix}(\sigma'). \]
	仍得到矛盾 $\left|\text{Fix}(\sigma')\right| > \left|\mathrm{Fix}(\sigma)\right|$. 明所欲证.
\end{proof}
\begin{corollary}
	当$n\geqslant5$时,$\mathfrak{A}_n$是$\mathfrak{S}_n$的唯一非平凡正规子群.
\end{corollary}

利用以上结果和Sylow定理可知最小非阿贝尔单群的阶数是60,且必同构于$\mathfrak{A}_n$.

\section{群列}
\begin{definition}\label{def:exact-seq-group}\index{zhenghelie@正合列 (exact sequence)}
	考虑一列群同态
	\[ \cdots \xrightarrow{f_0} G_1 \xrightarrow{f_1} G_2 \xrightarrow{f_2} \cdots \xrightarrow{f_i} G_{i+1} \to \cdots, \]
	长度或有限或无限. 若对所有 $i$ 都有
	\[ \Image(f_i) = \Ker(f_{i+1}), \]
	则称此列\textbf{正合}. 
	\begin{figure}[htp]
		\centering
		\includegraphics[scale=0.15]{Exact_Sequence_of_Groups}
		\caption{Illustration of an exact sequence of groups $G_{i}$ using Venn diagrams}
	\end{figure}
	我们经常把 $\{1\}$ 简写为 $1$, 或用加性符号记为 $0$. 举例明之, 对于任意同态 $\varphi: G \to G'$, 列 $G \to G' \to 1$ 正合当且仅当 $\varphi$ 是满的, 列 $1 \to G \to G'$ 正合当且仅当 $\varphi$ 是单的.\textbf{短正合列}为具有下列形式的正合列
	\[ 1 \to G' \xrightarrow{f} G \xrightarrow{g} G''  \to 1 \]
	如上所述,对任何一个短正合序列,$f$ 一定为单射,且$g$ 一定为满射,且$f$的像会等于$g$的核.有时也称$G$为$G''$经由$G'$的\textbf{扩张},亦即$G$可表为$G'\rtimes G''$.而若$G$可表为$G'\times G''$,则称为\textbf{平凡扩张};若$G'$落在$G$的中心,则称为\textbf{中心扩张}.
\end{definition}

正合列经常和交换图表搭配. 其妙用在同调代数中才会完全彰显,在Galois理论中将不会用到.
\begin{definition}\label{def:normal-series}\index{zhengguilie@正规列 (normal series)}\index{zhengguilie@正规列 (normal series)!次正规列 (subnormal series)}
	群 $G$ 的递降子群链
	\[ G = G_0 \geqslant G_1 \geqslant \cdots \geqslant G_n = \{1\} \]
	如满足 $\forall 0 \leq i < n,\;  G_{i+1} \lhd G_i$, 则称之为\textbf{次正规列},若还有$G_i \lhd G$则称为\textbf{正规列},而群族
	\[ G_i/G_{i+1}, \quad i=0, \ldots, n-1 \]
	称为该列的\textbf{因子群}. 正规列的\textbf{加细}是透过形如 \index{yinziqun@因子群 (factor subgroup)}
	\[ \left[ \cdots \rhd G_i \rhd G_{i+1} \rhd \cdots \right] \leadsto \left[ \cdots \rhd G_i \rhd G' \rhd G_{i+1} \rhd \cdots \right] \]
	的反复插项得到的新列. 插入 $G' = G_i$ 或 $G_{i+1}$ 得到的加细是平凡的; 反之则称为\textbf{真加细}.
\end{definition}
\begin{definition}\label{def:central-series}\index{zhongxinlie@中心列 (central series)}
	群 $G$ 的正规列 $G = G_0 \rhd G_1 \rhd \cdots$ 如对每个 $i$ 都满足
	\[
	G_i/G_{i+1} \leqslant Z_{G/G_{i+1}},
	\]
	则称为\textbf{中心列}.
\end{definition}
\begin{definition}\label{def:composition-series}\index{hechenglie@合成列 (composition series)}\index{zhuxulie@主序列 (chief series)}
	若群 $G$ 的次正规列 $G = G_0 \rhd G_1 \rhd \cdots$ 满足 $G_{i+1} \subsetneq G_i$, 而且因子群皆为单群, 则称之为\textbf{合成列},若同时为正规列,则称为\textbf{主序列}.
\end{definition}

细观单群定义可见合成列正是无冗余项, 而且无法再(真)加细的列. 有限群总有合成列, 一般的群则未必.
\begin{lemma}[Zassenhaus 蝴蝶引理]\label{prop:Zassenhaus}
	固定群 $G$, 考虑子群 $U, V$ 及各自的正规子群 $u \lhd U$, $v \lhd V$. 则有
	\begin{gather*}
		u(U \cap v) \lhd u(U \cap V), \\
		(u \cap V) v \lhd (U \cap V)v,
	\end{gather*}
	其中各项在注记 \ref{rem:HN} 的意义下都是子群, 而且有自然的同构
	\[ \dfrac{u (U \cap V)}{u (U \cap v)} \cong \dfrac{(U \cap V)v}{(u \cap V)v}. \]
\end{lemma}
\begin{proof}
	我们将表解各子群之间的关系, 图例如下:
	\[ \begin{tikzcd}
		G \arrow[dash, d] \\ H: \text{子群}
	\end{tikzcd} \quad
	\begin{tikzcd}
		G \arrow[dash, d, "\triangledown" description] \\ N: \text{正规子群}
	\end{tikzcd} \quad
	\begin{tikzcd}[column sep=tiny]
		A \arrow[dash, rd] & & B \arrow[dash, ld] \\
		& A \cap B &
	\end{tikzcd} \quad
	\begin{tikzcd}[column sep=tiny]
		{}& HN \arrow[dash, ld] \arrow[dash, rd, "\triangleleft" description] & \\
		H & & N
	\end{tikzcd}\]
	其中 $H \subset N_G(N)$. 现断言有以下图表:
	\[ \begin{tikzcd}[column sep=tiny]
		{} & U \arrow[dash, d] & & V \arrow[dash, d] & \\
		& u(U \cap V) \arrow[dash, dd, "\triangledown" description] \arrow[dash, rd] & & (U \cap V)v \arrow[dash, dd, "\triangledown" description] \arrow[dash, ld] & \\
		& & U \cap V \arrow[dash, dd, "\triangledown" description] & & \\
		& u(U \cap v) \arrow[dash, ld] \arrow[dash, rd] & & (u \cap V)v \arrow[dash, ld] \arrow[dash, rd] & \\
		u \arrow[dash, rd] & & (u \cap V)(U \cap v) \arrow[dash, ld] \arrow[dash, rd] & & v \arrow[dash, ld] \\
		& u \cap V & & U \cap v &
	\end{tikzcd} \]
	即证.故该引理也被称为蝴蝶引理.
\end{proof}
\begin{definition}\label{def:JH-subquotients}
	设 $G = G_0 \rhd \cdots$ 为次正规列, 我们视其因子群 $(G_i/G_{i+1})_{i \geq 0}$ 为不计顺序, 但计入重数的集合. 如果两个次正规列长度相同, 而且其因子群在上述意义下相等, 则称两次正规列\textbf{等价}.
\end{definition}

\begin{theorem}[Schreier 加细定理]\label{prop:Schreier}
	设
	\begin{align*}
		G & = G_0 \rhd \cdots \rhd G_r \rhd G_{r+1} = \{1\}, \\
		G & = H_0 \rhd \cdots \rhd H_s \rhd H_{s+1} = \{1\}
	\end{align*}
	为 $G$ 的两个次正规列, 则两者有等价的加细.
\end{theorem}
\begin{proof}
	对每个 $0 \leq i \leq r$, $0 \leq j \leq s$ 定义
	\begin{align*}
		G_{i,j} & \coloneqq G_{i+1} (H_j \cap G_i), \\
		H_{j,i} & \coloneqq (G_i \cap H_j) H_{j+1}.
	\end{align*}
	先看 $G_{i,j}$, 由 $G_{i+1} \lhd G_i$ 知其为子群. 包含关系 $G_{i, j+1} \lhd G_{i,j}$ 成立, 而且
	\[ G_{i,0} = G_{i+1} (G \cap G_i) = G_i, \quad G_{i,s+1} = G_{i+1}, \]
	遂得到 $(G_i)_{i=0}^r$ 的加细
	\[ \mathcal{G} \coloneqq \left[ \cdots \rhd G_i = G_{i, 0} \rhd G_{i, 1} \rhd \cdots G_{i, s} \rhd G_{i, s+1}= G_{i+1} \rhd \cdots \right] . \]
	同理可见 $H_{j, i}$ 给出 $(H_j)_{j=0}^s$ 的加细, 记为 $\mathcal{H}$. 在引理 \ref{prop:Zassenhaus} 中取 $u \coloneqq G_{i+1}$, $U \coloneqq G_i$ 和 $v \coloneqq H_{j+1}$, $V \coloneqq H_j$, 遂导出
	\[ \dfrac{G_{i,j}}{G_{i,j+1}} = \dfrac{u(U \cap V)}{u(U \cap v)} \cong \dfrac{(U \cap V)v}{(u \cap V)v} = \dfrac{H_{j,i}}{H_{j,i+1}}. \]
	
	当 $(i,j)$ 取遍所有可能, 次正规列 $\mathcal{G}$, $\mathcal{H}$ 的各个因子群在同构两边都恰好出现一次. 证毕.
\end{proof}

\begin{corollary}[Jordan--Hölder 定理]\label{cor:JH-group}
	群 $G$ 的任两个合成列皆等价.
\end{corollary}

因此, 一旦群 $G$ 有合成列, 则其因子群在定义 \ref{def:JH-subquotients} 的意义下无关合成列的选取.

\begin{definition}\index{daochulie@导出列 (derived series)}\index{zhongxinlie@中心列 (central series)!降中心列 (lower central series)}\index{zhongxinlie@中心列 (central series)!升中心列 (upper central series)}
	对于 $x, y \in G$, 定义\textbf{换位子}
	\[ [x,y] \coloneqq xyx^{-1}y^{-1}. \]
	对任意子集 $A, B \subset G$, 置 $[A, B] \lhd G$ 为 $\{[a,b] : a \in A, b \in B\}$ 的正规闭包.递归地定义
	\begin{enumerate}
		\item \textbf{导出列}: $G^{(0)} \coloneqq G, G^{(i)} \coloneqq [G^{(i-1)}, G^{(i-1)}](\forall i\geqslant1)$;
		\item \textbf{降中心列}: $G_1 \coloneqq G, G_i \coloneqq [G, G_{i-1}](\forall i\geqslant2)$;
		\item \textbf{升中心列}: $Z_0 \coloneqq \{1\},Z_{i} \coloneqq \{x\in G \mid \forall y\in G:[x,y] \in Z_{i-1} \}(\forall i\geqslant1)$.
	\end{enumerate}
\end{definition}
容易验证以下性质. 设 $i \in \mathbb{Z}_{\geq 0}$:
\begin{enumerate}
	\item $xy=yx \iff [x,y]=1$, 而 $[x,y]^{-1} = [y,x]$;
	\item 对于任意群同态 $\varphi: G_1 \to G_2$, 有 $\varphi [x,y] = [\varphi(x), \varphi(y)]$;
	\item $G^{(i)} \leqslant G_i$;
	\item $G^{(i)} \lhd G$, $G_i \lhd G$: 事实上 $G$ 的任何自同构都保持子群 $G^{(i)}$ 和 $G_i$.
	\item $G_i/G_{i+1} = Z_{G/G_{i+1}}$, $Z_i/Z_{i+1} = Z_{Z/Z_{i+1}}$.
\end{enumerate}

关于 $G^{(i)}$ 和 $G_i$ 的性质可以递归地证明. 我们也称 $G^{(1)}$ 为 $G$ 的\textbf{导出子群}或\textbf{换位子群}. 若$N\lhd G$,且$G/N$为阿贝尔群,则$G^{(1)}\leqslant N$,故 $G^\text{ab} \coloneqq G/G^{(1)}$ 称为 $G$ 的\textbf{交换化},$G$是阿贝尔群当且仅当$G^{(1)}=\{1\}$.\index{qun@群 (group)!子群 (subgroup)!导出子群 (derived subgroup)}

\begin{example}
	群 $\mathfrak{S}_n$ 的导出子群$\mathfrak{S}_n^{(1)}$ 等于 $\mathfrak{A}_n$. 当 $n \geqslant 5$ 时 $\mathfrak{A}_n$ 是非交换单群, 因此它必然等于自身的导出子群 $\mathfrak{A}_n^{(1)}$.
	
	当 $n=1$ 时此为显然. 以下解释 $n \geq 2$ 情形: $\mathfrak{S}_n$ 由对换生成, 每个对换都共轭于 $(1 \; 2)$, 故交换商 $\mathfrak{S}_n / \mathfrak{S}_n^{(1)}$ 由 $(1 \; 2)$ 的像生成, 这是二阶元. 给出商同态
	\[ \mathfrak{S}_n / \mathfrak{S}_n^{(1)} \twoheadrightarrow \mathfrak{S}_n/\mathfrak{A}_n \cong \{ \pm 1\}. \]
	比较阶数可见以上同态实为同构.
\end{example}
\section{可解群}
\begin{definition}\label{def:solvable-nilpotent-group}\index{qun@群 (group)!可解群 (solvable group)}\index{qun@群 (group)!超可解群 (supersolvable group)}\index{qun@群 (group)!幂零群 (nilpotent group)}\index{qun@群 (group)!多循环群 (polycyclic group)}
	设 $G$ 为群.给出如下等价定义:
	\begin{center}
		\begin{tabularx}{\textwidth}{cXX}
			\toprule
			&\textbf{可解群}&\textbf{幂零群}\\\midrule
			\romannumeral1)&导出列终止于$\{1\}$&降中心列终止于$\{1\}$\\
			\romannumeral2)&存在正规列使得每个因子群都交换&升中心列终止于$G$\\
			\romannumeral3)&存在次正规列使得每个因子群都交换&存在中心列\\
			\romannumeral4)&存在次正规列使得每个因子群都为素数阶循环群&存在正规列 $G = G_0 \rhd G_1 \rhd \cdots \rhd G_r = \{1\}$ 使得每个$[G,G_i]\leqslant G_{i+1}$\\
			\bottomrule
		\end{tabularx}
	\end{center}

	若存在正规列使得每个因子群都为素数阶循环群, 则称之为\textbf{超可解群};
	
	若存在次正规列使得每个因子群都为循环群, 则称之为\textbf{多循环群};
\end{definition}

设 $G$ 为幂零群, 则对任意 $x \in G$, 映射 $[x, \cdot]: g \mapsto [x, g]$ 迭代有限次次后的像落在 $\{1\}$, 故成为平凡映射 $g \mapsto 1$. 这解释了``幂零''一词的来由.而根据可解群的定义\romannumeral4),非素数阶单群是不可解的.
\begin{proposition}
	幂零蕴涵可解. 事实上,对于有限群,
	\[\text{循环群}\subset\text{阿贝尔群}\subset\text{幂零群}\subset\text{超可解群}\subset\text{多循环群}\subset\text{可解群}\subset\text{有限生成群}.\]
\end{proposition}
\begin{definition}\index{kejichengde@可继承的 (heritable)}
	性质$\mathcal{P}$称为\textbf{可继承的},若群 $G$ 具有性质 $\mathcal{P}$, 则 $G$ 的子群和商群都有性质 $\mathcal{P}$.
\end{definition}
\begin{lemma}\label{prop:solvable-ses}
	循环,阿贝尔,可解,超可解,幂零都是可继承的.
\end{lemma}

可解群的扩张亦是可解群;而幂零群的扩张未必是幂零群,但其中心扩张为幂零群.
\begin{proposition}
	设 $N \lhd G$,则 $G$ 可解当且仅当 $N,G/N$ 皆可解.
\end{proposition}

下述推论是证明五次以上方程无根式解的群论钥匙.
\begin{corollary}\label{prop:S_n-unsolvable}
	当 $n \geq 5$ 时 $\mathfrak{S}_n$ 不可解.
\end{corollary}


\begin{theorem}[Burnside $p^aq^b$定理]
	$p^aq^b$($p,q$是素数,$a,b$是正整数)阶群是可解群.
\end{theorem}

关于可解有限群最著名的结果当属英国数学家Burnside的猜想,该猜想于1963年由Walter Feit和John Griggs Thompson证明.
\begin{theorem}[Feit-Thompson 定理]\label{thm:Feit-Thompson}
	任意奇数阶有限群皆可解.
\end{theorem}

有限单群的分类是代数学里的一个巨大的工程.有关的文章大多发表于1955年至2004年之间,目的在于将所有的有限单群都给清楚地分类.这项工程总计约有100位作者在500篇期刊文章中写下了上万页的文字.该定理曾经有力地推动了有限单群的分类工作; 作为一篇有限群论的论文, 其255页的长度与繁复亦属空前, 然而还远远不是绝后的.
\begin{corollary}
	除素数阶循环群外,所有有限单群的阶都是偶数.
\end{corollary}
\section{多项式补遗}
\begin{definition}\index{huan@环 (ring)!多项式环 (polynomial ring)}\index{youlihanshuyu@有理函数域 (field of rational functions)}
	设$R$是环,$R$上以$X$为变元集的\textbf{多项式环}记作$R[X]$,包含如下元素
	\[ f = \sum_{a_1, \ldots, a_n \in \mathbb{N}\atop x_1,\dots,x_n\in X} c_{a_1, \ldots, a_n} x_1^{a_1} \cdots x_n^{a_n}. \]
	在其上有自然的加法和乘法.当$X=\{x,y,\dots\}$时,也记作$R[x,y,\dots]$.当$R$是交换环时,则称为多项式代数.设$K$是域,多项式环$F[X]$的分式域称为\textbf{有理函数域},记作$K(X)$,当$X=\{x,y,\dots\}$时,也记作$K(x,y,\dots)$.
	下标稍嫌繁杂, 我们顺势引进方便的多重指标符号,令$|X|=n$,
	\begin{align*}
		\bm{a} & \coloneqq (a_1, \ldots, a_n),\quad |\bm{a}| \coloneqq a_1 + \cdots + a_n,\quad c_{\bm{a}} \coloneqq c_{a_1, \ldots, a_n}, \\
		\bm{x}& \coloneqq(x_1,x_2,\dots,x_n),\quad \bm{x}^{\bm{a}} \coloneqq x_1^{a_1} \cdots x_n^{a_n}.\\
		&\implies f=\sum_{\bm{a}\in \mathbb{N}^n} c_{\bm{a}} \bm{x}^{\bm{a}}.
	\end{align*}
	
	注意到$R[x,y]=R[x][y]$.
\end{definition}
\begin{definition}\index{cishu@次数 (degree)}
	定义多项式 $f$ 的\textbf{次数}为 $\deg f \coloneqq \max\left\{ |\bm{a}| : |c_{\bm{a}}|\neq 0 \right\}$. 如果 $f$ 满足于 $c_{\bm{a}} \neq 0 \iff |\bm{a}|=m$, 则称 $f$ 是 $m$ 次\textbf{齐次多项式}.
\end{definition}
\begin{example}
	若$K$是域,在定义 \ref{def:Euclidean-ring} 中取$\deg$使得$K[X]$为ED.
\end{example}
\begin{example}\label{eg:Gauss-integers}
	Gauss 整数环定义为
	\[ \mathbb{Z}[\sqrt{-1}] \coloneqq \left\{ x+y\sqrt{-1} : x,y \in \mathbb{Z} \right\} \quad \text{(作为 $\mathbb{C}$ 的子环)}. \]
	在定义 \ref{def:Euclidean-ring} 中取范数映射
	\[ N(x+y\sqrt{-1}) = |x+y\sqrt{-1}|^2 = x^2 + y^2 \;\in \mathbb{N}. \]
	使得其为ED.
	
	
	注意到 $\mathbb{Z}[\sqrt{-1}]$ 对共轭运算 $z \mapsto \bar{z}$ 封闭, $N(z)=z\bar{z}$ 是乘法幺半群的同态, 由此不难推得 $\mathbb{Z}[\sqrt{-1}]^\times = N^{-1}(\mathbb{Z}^\times) = \{\pm 1, \pm\sqrt{-1}\}$.
\end{example}
\begin{proposition}
	若$R$是UFD/整环,则$R[X]$亦然.域上的一元多项式环为PID;环上则未必,如$\mathbb{Z}[\sqrt{5}].$
\end{proposition}
\begin{definition}\index{huan@环 (ring)!Noether环 (Noetherian ring)}
	称环$R$为\textbf{Noether环},若$R$的任一理想都是有限生成的.
\end{definition}
\begin{theorem}[Hilbert基定理]
	若$R$为Noether环,则$R[x]$亦然.
\end{theorem}

该定理与\textbf{Hilbert零点定理(Nullstellensatz)}相关,后者是代数几何中的基本定理.
\begin{definition}\index{duichengduoxiangshi@对称多项式 (symmetric polynomial)}
	称一个多项式$f\in R[X]$为\textbf{对称多项式},若$\forall \sigma\in \mathfrak{S}_X$, $f(\bm{x})=f(\sigma\bm{x})$.记所有的$n$元对称多项式构成的环为$\Lambda_n$.
\end{definition}
\begin{definition}\index{duichengduoxiangshi@对称多项式 (symmetric polynomial)!初等对称多项式 (elementary symmetric polynomial)}\label{def:ele-sym}
	定义$\bm{x}$的第$k$个$n$元\textbf{初等对称多项式}为
	\[e_k\coloneqq\sum_{1 \leq i_1 < \cdots < i_k \leq n} x_{i_1} \cdots x_{i_n}.\]
\end{definition}
\begin{theorem}[对称多项式基本定理]\label{thm:sym-basic}
	设$R$是环,$f\in \Lambda_n$是$R$上的对称多项式当且仅当是一些$n$元初等对称多项式的代数组合.即$\Lambda_n\cong R[e_1,e_2,\dots,e_n]$.
\end{theorem}
	\chapter{伽罗瓦理论}
本章开始介绍Galois理论.如无特殊说明,本章所讨论的Galois扩张均为有限扩张.详细的参考资料可参见\cite{Patrick}\cite{ZhP}.
\section{域扩张}
\begin{definition}\index{yukuozhang@域扩张 (field extension)}\index{yukuozhang@域扩张 (field extension)!有限生成扩张 (finitely generated extension)}\index{yukuozhang@域扩张 (field extension)!单扩张 (simple extension)}\index{benyuanyuan@本原元 (primitive element)}
	设$K$和$E$均为域,如果存在域同态$\iota:K\hookrightarrow E$,则称$(E,\iota)$为$K$的\textbf{域扩张},其中$K$是域扩张的\textbf{基域},$E$为$K$的\textbf{扩域},并记为$E/K$.若存在$S\subset E$,使得$F(S)=E$,则称$E$是由$S$在$K$上生成的域.若$S$为有限集,则称该扩张为\textbf{有限生成扩张};若$S=\{x\}$,则称该扩张为\textbf{单扩张},$x$称为\textbf{本原元}.
\end{definition}
\begin{definition}\index{cishu@次数 (degree)}\index{yukuozhang@域扩张 (field extension)!平凡扩张 (trivial extension)}\index{yukuozhang@域扩张 (field extension)!二次扩张 (quadratic extension)}\index{yukuozhang@域扩张 (field extension)!有限扩张 (finite extension)}\index{yukuozhang@域扩张 (field extension)!无限扩张 (infinite extension)}\label{def:2 ext}
	每个域扩张中,扩域$E$可以看作是以基域$K$为系数域的向量空间,称扩张$E/K$的\textbf{次数}为$[E:K]=\dim_K E$.次数为1的扩张称为\textbf{平凡扩张},此时$E$与$K$同构;次数为2的扩张称为\textbf{二次扩张},可以证明二次扩张必形如$E=K(\sqrt{d})$,从而是单扩张;次数有限的扩张称为\textbf{有限扩张},否则称为\textbf{无限扩张},取向量空间的一组基可知有限扩张必为有限生成扩张.
\end{definition}
\begin{definition}\index{yukuozhang@域扩张 (field extension)!子扩张 (subextension)}\label{def:telescope}
	若存在域扩张$E/F$和$F/K$,则称$F$为\textbf{中间域},$F/K$是$E/F$的\textbf{子扩张}.此时满足关系式$	[E:K]=[E:F][F:K]$,称作\textbf{望远镜公式}.
\end{definition}
\begin{theorem}[本原元定理]\label{thm:primitive element}
	一个有限扩张$E/K$是单扩张,即存在本原元$x\in E,E=K(x)$,当且仅当$E$和$F$之间有有限个中间域.
\end{theorem}
\begin{definition}\index{yukuozhang@域扩张 (field extension)!代数扩张 (algebraic extension)}\index{yukuozhang@域扩张 (field extension)!超越扩张 (transcendental extension)}
	对于域扩张$E/K$,如果$a\in E$是$K$上非零多项式的根,则称$a$在$K$上\textbf{代数},否则称$a$在$K$上\textbf{超越}.若$E$中所有元素均在$K$上代数,则$E/K$称为\textbf{代数扩张},否则称为\textbf{超越扩张}.
\end{definition}
\begin{definition}\index{jixiaoduoxiangshi@极小多项式 (minimal polynomial)}\label{def:minimal polynomial}
	若$a$在$K$上代数,则存在唯一一个次数最小的首一多项式,称为$a$在$K$上的\textbf{极小多项式}$m_a$,则有同态映射$\pi:K[a]\to K(a)$,且$\Ker(\pi)=(m_a)$,故由同态基本定理知
	\[
	K[a]/(m_a)\cong K(a).
	\]
	由此可见以$\{1,a,a^2,\dots,a^{n-1}\}$为基的$K$-线性空间即为$K(a)$,故$[K(a):K]=\deg m_a$.故代数扩张为有限扩张,反之,$\{1,a,a^2,\dots,a^{n-1},a^n\}$必线性相关,故存在$a$的化零多项式,故有限扩张也为代数扩张.因此有限扩张无非是有限生成的代数扩张.
\end{definition}
\begin{definition}\index{yu@域 (field)!代数闭域 (algebraically closed field)}\index{daishubibao@代数闭包 (algebraic closure)}
	域$K$称为\textbf{代数闭域},若$a$在$K$上代数蕴含$a\in K$.代数扩张$E/K$ 若满足$E$为代数闭域,则称之为 $K$ 的\textbf{代数闭包},并记$E$为$\overline{K}$或$K^{\text{alg}}$.
\end{definition}
\begin{theorem}[代数基本定理]
	$\mathbb{C}$是代数闭包.
\end{theorem}
\section{正规扩张与可分扩张}
\begin{definition}\index{yu@域 (field)!分裂域 (splitting field)}
	设$E/K$为域扩张,称多项式$p\in K[x]$在$E$上\textbf{分裂},若其在$E[x]$可分解为一次因子的积.设 $\mathcal{P}$ 为 $K[x]$ 中一族非常数多项式. 若域扩张 $E/K$ 满足于
	\begin{enumerate}
		\item 每个 $p \in \mathcal{P}$ 皆在 $E$ 上分裂;
		\item 诸根 $R\coloneqq\left\{ \alpha_{p,j} : p \in \mathcal{P},\; 1 \leq j \leq \deg p \right\}$ 在 $K$ 上生成 $E$,即$E=K(R)$.
	\end{enumerate}
	则称 $E/K$ 为多项式族 $\mathcal{P}$ 的\textbf{分裂域}.
\end{definition}
\begin{proposition}
	设$f\in K[x],\deg f\geqslant1$,则$f$在$K$上的分裂域$E/K$存在且唯一,且$[E:K]\leqslant(\deg f)!$.
\end{proposition}
\begin{definition}\label{def:normal-ext}\index{yukuozhang@域扩张 (field extension)!正规扩张 (normal extension)}
	对于代数扩张 $E/K$, 以下性质等价:
	\begin{enumerate}
		\item 任一不可约多项式 $p \in K[x]$ 若在 $E$ 中有根, 则它在 $E$上分裂;
		\item 取定代数闭包 $\overline{K}/E$ 并视 $E$ 为 $\overline{K}$ 的子域, 则任意 $\iota \in \Hom_K(E, \overline{K})$ 皆满足 $\iota(E)=E$.
		\item 存在一族非常数多项式 $\mathcal{P}$ 使得 $E/K$ 是 $\mathcal{P}$ 的分裂域.
	\end{enumerate}
	满足以上任一条的代数扩张称为\textbf{正规扩张}.
\end{definition}
\begin{definition}\label{def:normal-closure}\index{zhengguibibao@正规闭包 (normal closure)}
	设 $E/K$ 为代数扩张, $\overline{K}/E$ 为选定的代数闭包. 定义 $E/K$ 的\textbf{正规闭包} $M/K$ 为 $\overline{K}|K$ 中所有含 $E/K$ 的正规子扩张之交.即
	\[
	M=\bigcap_{E\leqslant N\leqslant \overline{K}\atop N/K\text{正规}}N.
	\]
	由定义知这是包含$E$的$F$之最小正规扩张.
\end{definition}
\begin{example}\label{eg:x3-2-splitting}
	$\mathbb{Q}(\sqrt[3]{2})$的正规闭包是$\mathbb{Q}(\sqrt[3]{2},\omega)$,其中$\omega$是三次单位根.
\end{example}
\begin{definition}\index{kefenduoxiangshi@可分多项式 (separable polynomial)}\index{yukuozhang@域扩张 (field extension)!可分扩张 (separable extension)}\index{yukuozhang@域扩张 (field extension)!纯不可分扩张 (purely inseparable extension)}
	称非常数多项式$p\in K[x]$为\textbf{可分多项式},若其无重根.对于代数扩张$E/K$,若元素$a\in E$在$K$上的最小多项式为可分多项式,则称$a$在$K$上\textbf{可分}.记$E$中所有可分元素为$E_s$,此为$E/K$的中间域,则称$[E:K]_s\coloneqq[E_s:K]$为\textbf{可分次数},若$E/K$为有限扩张,则称$[E:K]_i\coloneqq[E:K]/[E_s:K]$为\textbf{不可分次数}.如果$E$中每个元素都在$K$中可分,即不可分次数为1,则称$E/K$为\textbf{可分扩张}.若可分次数为1,则称$E/K$为\textbf{纯不可分扩张}.
\end{definition}
\begin{definition}\index{kefenbibao@可分闭包 (separable closure)}
	对于$K$的代数闭包$\overline{K}$中的所有可分元素构成中间域$K^{\text{sep}}$,称为\textbf{可分闭包}.
\end{definition}
\begin{corollary}
	由本原元定理\ref{thm:primitive element},有限可分扩张必是单扩张.
\end{corollary}
\begin{definition}\label{def:perfect-field}\index{yu@域 (field)!完全域 (perfect field)}
	若域 $F$ 的所有代数扩张都是可分扩张, 也即$F[x]$中的每个不可约多项式都是可分多项式,则称 $F$ 为\textbf{完全域}.
\end{definition}
\begin{example}
	特征零域均为完全域;特征$p$域$K$是完全域当且仅当$K=K^p\coloneqq \{a^p|a\in K\}$,故有限域均为完全域.
\end{example}

\section{伽罗瓦扩张}
\begin{definition}\index{{qun@群 (group)!Galois群 (Galois group)}}\index{yu@域 (field)!不动域 (invariant field)}
	对于域扩张$E/K$,其\textbf{Galois群}定为 $\Gal(E/K) \coloneqq \Aut_K(E)$,表示保持$K$不动的$E$的自同构群.习惯用 $\Gal(E/K)$ 的群论性质来描述 $E/K$, 例如称 $E/K$ 为交换 (或循环, 可解等) 扩张, 如果 $\Gal(E/)$ 作为群是交换的 (或循环, 可解等).
	
	另一方面,若$H$为$\Aut(E)$的子群,则其\textbf{不动域}定为$\Inv(H)\coloneqq E^H$.故而有两个映射
	\begin{center}
		\begin{tabular}{rccl}
			$\Gal(E/\cdot)$&:&$E$的子域$\to\Aut(E)$的子群,&$K\mapsto \Gal(E/K)$\\
			$\Inv$&:&$\Aut(E)$的子群$\to E$的子域,&$H\mapsto\Inv(H)$\\
		\end{tabular}
	\end{center}
\end{definition}
\begin{proposition}
	设$E/K$是任意域扩张, $G=\Gal(E/K)$,则
	\begin{enumerate}
		\item $\Gal(E/\cdot)$和$\Inv$是反序的映射,即
		
		若$M_1$和$M_2$均是$E/K$的中间域,且$M_1\subset M_2$,则$\Gal(E/M_1)\geqslant\Gal(E/M_2)$;
		
		若$H_1$和$H_2$均是$G$的子群,且$H_1\leqslant H_2$,则$\Inv(H_1)\supset\Inv(H_2)$.
		\item 对于中间域$M$有$M\subset\Inv(\Gal(E/M))$;对于$G$的子群$H$有$H\leqslant\Gal(\Inv(H))$.
		\item 对于中间域$M$有$\Gal(E/M)=\Gal(E/\Inv(\Gal(E/M)))$;对于$G$的子群$H$有$\Inv(H)=\Inv(\Gal(\Inv(H)))$.
	\end{enumerate}
\end{proposition}

利用线性代数的知识我们可得以下两条有用的引理:
\begin{lemma}[Artin]
	设$H$是域$K$的自同构群的有限子群,则$[K:\Inv(H)]\leqslant|H|$.
\end{lemma}
\begin{lemma}
	设$E/K$是有限扩张,则$|\Gal(E/K)|\leqslant[E:K]$.
\end{lemma}
\begin{definition}\label{def:Galois-ext} \index{yukuozhang@域扩张 (field extension)!Galois扩张 (Galois extension)}
	设$E/K$是有限扩张,以下性质等价:
	\begin{enumerate}
		\item $E$是$K[x]$中一族可分多项式的分裂域;
		\item $E/K$是正规可分扩张;
		\item $\Gal(E/K)=[E:K]$或$[E:\Inv(H)]=|H|$,其中$H\leqslant\Gal(E/K)$;
		\item $K=\Inv(\Gal(E/K))$或$H=\Gal(E/\Inv(H))$,其中$H\leqslant\Gal(E/K)$;
		\item $E/K$是可分扩张,$F/E$是代数扩张,则$\forall\sigma\in\Gal(F/K),\sigma(E)=E$.
	\end{enumerate}
	满足以上任一条的扩张称为\textbf{(有限) Galois扩张}.
\end{definition}
\begin{definition}\index{yukuozhang@域扩张 (field extension)!复合 (compositum)}
	设 $E/K$ 为域扩张, $(E_i/K)_{i \in I}$ 为其中一族子扩张, 定义其\textbf{复合} $\bigvee_{i \in I} E_i$ 为 $E$ 中包含所有 $E_i$ 的最小域, 其元素形如有理分式
	\[
	\frac{P(x_{i_1}, \dots, x_{i_n})}{Q(x_{i_1}, \dots, x_{i_n})}, \quad Q(x_{i_1}, \dots, x_{i_n}) \neq 0,
	\]
	其中$n \geqslant 0$, $P, Q \in K[x_1, \dots, x_n]$, $i_1, \dots, i_n \in I$, $x_{i_k} \in E_{i_k}$.两个扩张的复合也写作 $E_1 E_2$.
\end{definition}
\begin{proposition}
	令$\mathcal{E}$分别为有限,代数,正规,可分, Galois,则
	\begin{enumerate}
		\item 对于扩张$E/F$,$F/K$, $E/K$是$\mathcal{E}$当且仅当$E/F$和$F/K$都是$\mathcal{E}$;(对于正规扩张例外)
		\item 扩张$E/K$的任意子扩张$F_1/K$和$F_2/K$,若$F_1/K$为$\mathcal{E}$,则$F_1F_2/F_2$也为$\mathcal{E}$.
		\item 扩张$E/K$的任意一族$\mathcal{E}$子扩张的复合和非空交仍为$\mathcal{E}$.(对于有限扩张例外,需要子扩张族有限)
	\end{enumerate}
\end{proposition}
\begin{theorem}[Galois理论基本定理]\label{thm:finite-Galois-corr}
	设$E/K$是有限Galois扩张, $G=\Gal(E/K)$,则
	\begin{enumerate}
		\item $\Gal(E/\cdot)$和$\Inv$是\textbf{互逆的}反序的映射.
		\item $G$的子群是共轭的当且仅当它们的不动域是共轭的,是正规的当且仅当其不动域是$K$的正规扩张.
		\item 设$F$为中间域,则有$\Gal(E/K)/\Gal(E/F)\cong\Gal(F/K)$.
	\end{enumerate}
\end{theorem}

Galois理论基本定理的作用是将域扩张的中间域结构,转化为特定群的子群来描述.将难以用直接的方法刻画的中间域,和可以用群论中的成熟方法刻画的有限群子群对应起来.下面介绍一个经典的例子.

\begin{example}
	重拾例 \ref{eg:x3-2-splitting} 中 $x^3-2 \in \mathbb{Q}[x]$ 的分裂域 $\mathbb{Q}(\sqrt[3]{2}, \omega)$, 注意到 $\omega$ 在 $\mathbb{Q}$ 上的极小多项式为 $x^2+x+1$. 绘制域图:
	\[\begin{tikzcd}[column sep=small]
		{} & \mathbb{Q}(\sqrt[3]{2}, \omega) \arrow[dash, d, "\text{Galois}"] \arrow[dash, ld, "\text{Galois}"'] \\
		\mathbb{Q}(\sqrt[3]{2}) \arrow[dash, rd, "\text{次数} = 3"'] & \mathbb{Q}(\omega) \arrow[dash, d, "{\text{Galois},\text{次数} = 2}"] \\
		& \mathbb{Q}
	\end{tikzcd}\]
	因为 $[\mathbb{Q}(\sqrt[3]{2}, \omega): \mathbb{Q}(\omega)] \leq 3$, 而 $2 = [\mathbb{Q}(\omega) : \mathbb{Q}]$ 和 $3 = [\mathbb{Q}(\sqrt[3]{2}) : \mathbb{Q}]$ 都得整除 $[\mathbb{Q}(\sqrt[3]{2}, \omega): \mathbb{Q}]$, 唯一的可能是 $[\mathbb{Q}(\sqrt[3]{2}, \omega): \mathbb{Q}] = 6$. 因之 $G := \Gal(\mathbb{Q}(\sqrt[3]{2},\omega)|\mathbb{Q})$ 是 $6$ 阶群. 任意 $\sigma \in G$ 的作用方式为 $\omega \mapsto \omega^{\pm 1}$, $\sqrt[3]{2} \mapsto \omega^k \sqrt[3]{2}$ ($k=0,1,2$), 并且 $\sigma$ 完全由 $(\pm, k)$ 确定, 至多有 $6$ 种选取, 于是每组 $(\pm,k)$ 都能在 $G$ 中实现.
	
	一般来说, 可分多项式的分裂域 的 Galois 群能嵌入为根集的对称群; 既然 $|G|=6=3!$, 现在可以等同 $G$ 与根集 $\{ \sqrt[3]{2}, \omega\sqrt[3]{2}, \omega^2\sqrt[3]{2} \}$ 上的对称群 $\mathfrak{S}_3$. 列举子群并考量这些子群所固定的元素, 应用定理 \ref{thm:finite-Galois-corr} 立得反序对应:
	\[\begin{tikzpicture}
		[baseline = (current bounding box.center),
		every node/.style={ },
		every edge/.style={draw},
		normaledge/.style={ sloped, edge node={node[fill=white, outer sep=0.2mm, inner sep=0mm] {$\lhd$}} }
		]
		\begin{scope}
			\node (G) at (0, 3) {$\mathfrak{S}_3$};
			\node (extra) at (0.8,3) {$G$};
			\draw (G) edge[draw=none, edge node={node {$\eqqcolon$}}] (extra);
			\node (A) at (-1, 2) {$\mathfrak{A}_3$};
			\node (H1) at (-2,1.5) {$\langle(1 2)\rangle$};
			\node (H2) at (0,1.5) {$\langle(2 3)\rangle$};
			\node (H3) at (1.5,1.5) {$\langle(1 3)\rangle$};
			\node (E) at (0, 0) {$\{1\}$};
		\end{scope}
		\draw (G) edge[normaledge] (A);
		\draw (G) edge[bend right=30] (H1);  \draw (G) edge (H2);  \draw (G) edge[sloped, edge label={$\supset$}] (H3);
		\foreach \x in {A, H1, H2, H3}
		\draw (\x) edge (E);
	\end{tikzpicture} 
	\begin{tikzpicture}[baseline= (current bounding box.center)]
		\draw[thick, dashed] (0, 1.7) -- (0, -1.7);
		\node[draw, fill=white] at (0,0) {$H\leftrightarrow \Inv(H)\atop \Gal(E/K)\leftrightarrow K$};
	\end{tikzpicture}  \begin{tikzpicture}
		[baseline = (current bounding box.center),
		every node/.style={ },
		every edge/.style={draw},
		normaledge/.style={edge node={node[fill=white, outer sep=0.2mm, inner sep=0mm] {\scriptsize$\text{Galois}$}} }
		]
		\begin{scope}
			\node (FG) at (0,3) {$\mathbb{Q}$};
			\node (FA) at (-1, 2) {$\mathbb{Q}(\omega)$};
			\node (FH1) at (-2, 1.5) {$\mathbb{Q}(\omega^2 \sqrt[3]{2})$};
			\node (FH2) at (0,1.5) {$\mathbb{Q}(\sqrt[3]{2})$};
			\node (FH3) at (1.5,1.5) {$\mathbb{Q}(\omega\sqrt[3]{2})$};
			\node (FE) at (0,0) {$\mathbb{Q}(\omega, \sqrt[3]{2})$};
			\node (extra) at (-1.4,0) {$E$};
			\draw (FE) edge[draw=none, edge node={node {$\coloneqq$}}] (extra);
		\end{scope}
		\draw (FG) edge[normaledge] (FA);
		\draw (FG) edge[bend right=30] (FH1);  \draw (FG) edge (FH2);  \draw (FG) edge[sloped, edge label={$\subset$}] (FH3);
		\foreach \x in {A, H1, H2, H3}
		\draw (F\x) edge (FE);
	\end{tikzpicture}\]
	
	对应的子群和中间域置于相同位置, 并以连线表示包含关系, 两侧的包含关系是上下颠倒的. 三个中间域 $\mathbb{Q}(\omega^k \sqrt[3]{2})$ ($k=0,1,2$) 两两共轭, 这从群论一面看应该是明显的.
\end{example}
	\chapter{尺规作图}
尺规作图(英语:compass-and-straightedge 或 ruler-and-compass construction)是起源于古希腊的数学课题.只使用圆规和直尺,并且只准许使用有限次,来解决不同的平面几何作图题.

值得注意的是,以上的``直尺''和``圆规''是抽象意义的,跟现实中的并非完全相同,具体而言,有以下的限制:
\begin{itemize}
	\item 直尺必须没有刻度,无限长,只可以做过两点之直线.
	\item 圆规可以开至无限宽,但上面亦不能有刻度.它只可以拉开成你之前构造过的长度或一个任意的长度.
\end{itemize}

尺规作图的研究,促成数学上多个领域的发展.有些数学结果就是为解决古希腊三大名题而得出的副产品,对尺规作图的探索推动了对圆锥曲线的研究,并发现了一批著名的曲线.

若干著名的尺规作图已知是不可能的,例如``尺规作图三大难题'':
\begin{enumerate}
	\item \textbf{三等分角(angle trisection)};
	\item \textbf{倍立方(doubling the cube/Delian problem)};
	\item \textbf{化圆为方(squaring the circle)}.
\end{enumerate}

而当中很多不可能的例子是利用了19世纪出现的Galois理论以证明.尽管如此,仍有很多业余者尝试这些不可能的题目.
\section{规矩数}
\begin{definition}\index{guijushu@规矩数 (constructable number)}
	称平面中可用尺规作图的方式作出的点为\textbf{规矩点}\footnote{[矩]字的繁体字榘,是矢字,巨字和木字组成.矢字代表短尺,巨是指巨大,木是指用木制作的尺,榘是用来量方的尺.故而``规矩点''即为用尺规构造出的点.}.为确定原点$O$和单位距离1,可事先给定平面中的两点并记为$\{(0,0),(1,0)\}$,则称由此生成的规矩点的横纵坐标表示的数为\textbf{规矩数},又称\textbf{可造数}.由于尺规可以在给定的坐标系中做投影映射,故规矩数又为尺规作图中圆规可以丈量长度的数.
\end{definition}

容易验证,有限个规矩数相加/减/乘/除(除数不得为0)仍为规矩数,故所有的规矩数构成了一个域,而这个域包含$\mathbb{Q}$.同时,一个规矩数的二次方根也为规矩数.

事实上,规矩数仅能完成以上五种操作,被称为尺规作图公法,下面我们来简单复现这些操作:

对于加减法是显然的,乘除法则需用到相似三角形,如下图所示,以$(1,0)$为垂足画长度为$a$的线段得到$A(1,a)$,再以$(b,0)$为垂足画垂线交$OA$的延长线于$B(b,ab)$点,即得到$ab$.除法则是相反操作,不赘述.

\begin{center}
	\begin{tikzpicture}[scale=2]
		\coordinate[label=below left:$O$] (O) at (0,0);
		\coordinate[label=below:$1$] (A1) at (1,0);
		\coordinate[label=above left:$A$] (A) at (1,0.7);
		\coordinate[label=below:$b$] (B1) at (2,0);
		\coordinate[label=above:$B$] (B) at (2,1.4);
		\draw 
		(O) to (B1) to (B) to (O)
		(A1) to (A)
		pic [draw,thick,red,scale=0.5] {right angle = O--A1--A}
		pic [draw,thick,red,scale=0.5] {right angle = O--B1--B};
	\end{tikzpicture}
\end{center}

对于开根号则需用到射影定理,对于如下的图形,我们有
\[AC^2=OA\cdot AB,\]
故只需取$OA=1,AB=a$,则有$AC=\sqrt{a}$.
\begin{center}
	\begin{tikzpicture}[scale=2]
		\coordinate[label=below left:$O$] (O) at (0,0);
		\coordinate[label=below:$A$] (A) at (1,0);
		\coordinate[label=below right:$B$] (B) at (2.41,0);
		\coordinate[label=left:$C$] (C) at (1,1.2);
		\draw 
		(O) to (B) to (C) to (O)
		(A) to (C)
		pic [draw,thick,red,scale=0.5] {right angle = O--A--C}
		pic [draw,thick,red,scale=0.5] {right angle = O--C--B};
	\end{tikzpicture}
\end{center}

于是利用二次扩张的定义\ref{def:2 ext}和望远镜公式\ref{def:telescope}和立刻得到如下引理.
\begin{lemma}\label{lem:2^s}
	任何规矩数$r$对应的域扩张扩张$\mathbb{Q}(r)/\mathbb{Q}$的次数都是2的方幂,即
	\[
	[\mathbb{Q}(r):\mathbb{Q}]=2^s,\quad s\in \mathbb{N}.
	\]
\end{lemma}
\section{三大难题}
\subsection{三等分角}
\begin{quotation}
	\textit{能否用尺规三等分任意角?}
\end{quotation}

显然,角$\theta$能否被尺规作出取决于$\cos \theta$是否为规矩数.不妨设$\cos3\theta$为规矩数,由三倍角公式
\[
\cos 3\theta=4\cos^3\theta-3\cos\theta,
\]
即$\cos\theta$多项式$f(x)=4x^3-3x-\cos3\theta$的根,当$f$在$\mathbb{Q}(\cos3\theta)$上不可约时,有
\[
[\mathbb{Q}(\cos\theta,\cos3\theta):\mathbb{Q}]=[\mathbb{Q}(\cos\theta,\cos3\theta):\mathbb{Q}(\cos3\theta)][\mathbb{Q}(\cos3\theta):\mathbb{Q}]=3\cdot2^s,
\]
于是由引理\ref{lem:2^s}和\ref{def:minimal polynomial}知$\cos\theta$非尺规数.所以能三等分角当且仅当$f$在$\mathbb{Q}(\cos3\theta)$上可约.

下面我们只需要找到一个$\cos3\theta$使得$f$在$\mathbb{Q}(\cos3\theta)$上不可约.这好办,取$\theta=\pi/18$.
\subsection{倍立方}
\begin{quotation}
	\textit{能否用尺规作一立方体的棱长,使其体积等于一给定立方体的两倍?}
\end{quotation}

若给定立方体的棱长为$a$,作出立方体的棱长$b$是多项式$f(x)=x^3-2a^3$,当这个多项式在$\mathbb{Q}(a)$上不可约时,与之同理有$b$不是规矩数.所以能倍立方当且仅当$f$在$\mathbb{Q}(a)$上可约.

下面我们只需要找到一个$a$使得$f$在$\mathbb{Q}(a)$上不可约.这更好办,取$a=1$.
\subsection{化圆为方}
\begin{quotation}
	\textit{能否用尺规作一正方形,其面积等于一给定圆面积?}
\end{quotation}

若给定正方形的边长为尺规数$a$,作出立方体的棱长$b$是多项式$f(x)=x^2-\pi a^2$,所以
\[
[\mathbb{Q}(b):\mathbb{Q}]=[\mathbb{Q}(b):\mathbb{Q}(a)][\mathbb{Q}(a):\mathbb{Q}]=[\mathbb{Q}(\sqrt{\pi}):\mathbb{Q}]2^s=[\mathbb{Q}(\pi):\mathbb{Q}]2^{s+1}=\infty,
\]
于是由引理\ref{lem:2^s}和\ref{def:minimal polynomial}知$b$非尺规数.

\begin{remark}
	这里需用到Lindemann–Weierstraß定理来说明$\pi$的超越性.
\end{remark}
\section{正多边形}
\begin{quotation}
	\textit{能否用尺规作正$n$边形?}
\end{quotation}

该问题与三大难题并列, Gauss (1777-1855)十九岁时解决了这个问题, 1801年他又给出了正十七边形的构造方法.

为解决这个问题,我们需要扩充规矩数的定义.
\begin{definition}\index{guijushu@规矩数 (constructable number)!复规矩数 (complex constructable number)}
	称$z\in\mathbb{C}$为\textbf{复规矩数},若其在复平面上位于规矩点.
\end{definition}

可以证明,复规矩数的许多性质和规矩数是一样的.

而为了作出正$n$边形,则需$n$次本原单位根$\zeta_n=\mathrm{e}^{2\pi\mathrm{i}/n}$是复规矩数.而$\zeta_n$在$\mathbb{Q}$上的极小多项式为分圆多项式$\Phi_n(x)$,而$\deg\Phi_n=\varphi(n)$.由引理\ref{lem:2^s}和\ref{def:minimal polynomial}可知能作出正$n$边形当且仅当$\varphi(n)$是2的方幂.

利用数论中的知识,对于素数分解$n=2^kp_1^{k_1} p_2^{k_2} \cdots p_r^{k_r}$,其中$p_1,p_2,\dots,p_r$为奇素数,有
\[
\varphi(n)=2^{k-1}p_1^{k_1-1} p_2^{k_2 - 1} \cdots p_r^{k_r - 1} (p_1 - 1) (p_2 - 1) \cdots (p_r - 1),
\]
故$\varphi(n)$为2的方幂当且仅当$k_1=k_2=\cdots=k_r=1$,且$p_1,p_2,\dots,p_r$为Fermat素数,即形如$1+2^s,s>0$.由$n$次方和公式知这里$s$无大于1的奇数因子,故Fermat素数必形如$F_n=1+2^{2^n},n\geqslant0$.前5个Fermat素数为:
\[
F_0=3,\quad F_1=5,\quad F_2=17,\quad F_3=257,\quad F_4=65537,
\]
但$F_5$和$F_6$不再是素数.

但请注意,这里只得到了可尺规作出正$n$边形的充分条件, Gauss认为这个条件也是必要条件,但是他一直没有发表他的证明. Pierre Wantzel 于1837年给出了一份完整的必要性的证明,因此这个定理被叫做 Gauss–Wantzel 定理.

为证必要性,我们需要下述引理.
\begin{lemma}\label{lem:2^s2}
	$z\in\mathbb{C}$是复规矩数当且仅当$z$属于$\mathbb{Q}$的$2^s(s\geqslant 0)$次正规扩张.
\end{lemma}

因为$\mathbb{Q}(i,\zeta_n)$是$(x^2+1)(x^n-1)$在$\mathbb{Q}$上的分裂域,故$\mathbb{Q}(i,\zeta_n)/\mathbb{Q}$是正规扩张,又$[\mathbb{Q}(\zeta_n):\mathbb{Q}]=\varphi(n)$是2的方幂,故$[\mathbb{Q}(i,\zeta_n):\mathbb{Q}]$是2的方幂.即满足引理\ref{lem:2^s2}的条件,于是我们得到了:
\begin{theorem}[Gauss–Wantzel]
	正$n$边形能被尺规作出当且仅当$n$是2的方幂和任意个(可为0个)相异费马素数的乘积.
\end{theorem}
	\chapter{方程的根式解问题}
Abel-Ruffini定理是代数学中的重要定理.它指出,五次及更高次的多项式方程没有一般的求根公式,即不是所有这样的方程都能由方程的系数经有限次四则运算和开方运算求根. Lagrange首先预见了高次方程的求根公式不存在,但始终无法证明; Paul Ruffini前者在1799年给出了一个不完整的证明,这个证明有严重错误; Niels Abel则在1824年给出了完整的证明. 

Galois在此基础上得到了更深刻的结果,并且就此问题引入的概念和方法,从推动学科发展的意义上说,兼具革命性和建设性.在计算机科学日益发展的今天,我们对高次方程的根式解并不过分关心,但Galois理论推动的在其它领域的建设依旧不可估量.
\section{方程的伽罗瓦群}
\begin{definition}
	设$E$是域$K$上$n$次多项式$f$在$K$上的分裂域,其$n$个根记为$r_1,r_2,\dots,r_n$,故$E=K(r_1,\dots,r_n)$,将$\Gal(E/K)$称为多项式$f$或方程$f(x)=0$在$K$上的Galois群,并记作$\Gal(f,K)$,或简写为$\Gal(f)$.则每个$\sigma\in\Gal(f)$是根集$R\coloneqq\{r_1,r_2,\dots,r_n\}$上的置换,故$\Gal(f)$可看作$\mathfrak{S}_n$的子群在$R$上的作用.当$f$可分时则$E/K$为Galois扩张.
\end{definition}

事实上,若$\sigma\in\mathfrak{S}_n$,则$\sigma\in\Gal(f)$当且仅当$\sigma$保持$R$之间的所有代数关系,即若
\[
g(r_1,r_2,\dots,r_n)=0,\forall g\in K[x_1,x_2\dots,x_n],
\]
则$g(\sigma(r_1),\sigma(r_2),\dots,\sigma(r_n))=0$.这也是Galois本人在\cite{Galois}中对$\Gal(f)$的定义,这反映了根集$R$之间的对称.今天所采用的定义来自于Dedekind,这一突出贡献不仅使得Galois理论取得了现代形式,也使Galois群的计算更具操作性.
\begin{definition}\index{panbieshi@判别式 (discriminant)}
	设非零多项式$f=a_nx^n+\cdots+a_1x+a_0$在$K$上分裂,其$n$个根记为$r_1,r_2,\dots,r_n$,记$A$为这些根所对应的Vandermonde矩阵,记$f$的Vandermonde行列式为
	\[
	|A|=\prod_{i<j}\left(r_i-r_j\right),
	\]
	则其\textbf{判别式}定义为
	\[
	D(f)=a_n^{2n-2}|A|^2=a_n^{2n-2}|A^TA|=a_n^{2n-2}\prod_{i<j}\left(r_i-r_j\right)^2,
	\]
	显然$D(f)=0$当且仅当$f$可分.
\end{definition}

更多内容请参考\cite[\S 4.5]{LJ}.
\begin{lemma}
	设域$K$的特征不为2, $f$是$K$上的首一$n$次可分多项式, $E$为$f$在$K$上的分裂域,由于偶置换不改变$f$的Vandermonde行列式的符号,故
	\[
	\Inv(\Gal(f)\cap\mathfrak{A}_n)=K(\sqrt{D(f)}),
	\]
	于是
	\[
	\Gal(f)\subseteq \mathfrak{A}_n\Longleftrightarrow \sqrt{D(f)}\in K\Longleftrightarrow D(f)\in K^2\coloneqq\{x|x^2\in F\}.
	\]
\end{lemma}
\begin{lemma}
	$f$是域$K$上的首一$n$次可分多项式, $E$为$f$在$K$上的分裂域, $f$在$E$上的$n$个根记为$r_1,r_2,\dots,r_n$则$f$在$K$上不可约$\Longleftrightarrow \Gal(f)$在$\{r_1,r_2,\dots,r_n\}$上传递,即为$\mathfrak{S}_n$的传递子群,阶为$[E:K]$.
\end{lemma}

对于$n=3$.由于$\mathfrak{S}_3$的传递子群仅有$\mathfrak{S}_3$和$\mathfrak{A}_3$,故$D(f)\notin F^2\Longleftrightarrow \Gal(f)=\mathfrak{S}_3$.

对于$n=4$. $\mathfrak{S}_4$的传递子群仅有$\mathfrak{S}_4,\mathfrak{A}_4,D_8,C_4,W=\{1,(12)(34),(13)(24),(14)(23)\}$,而$\Gal(f)$的具体分析较为复杂,可参照\cite[p.160]{FK}.

对于$n>4$的多项式, $f$的Galois群计算较为困难.
\section{伽罗瓦群的计算}
\begin{lemma}\label{lem:sym-gal}
	设$K$是域, $x_1,x_2,\dots,x_n$是独立的不定元, $e_1,e_2,\dots,e_n$是$x_1,x_2,\dots,x_n$的初等对称多项式(见\ref{def:ele-sym}),则域扩张$K(x_1,x_2,\dots,x_n)/K(e_1,e_2,\dots,e_n)$是有限Galois扩张,且
	\[
	\Gal(K(x_1,x_2,\dots,x_n)/K(e_1,e_2,\dots,e_n))=\mathfrak{S}_n;\qquad\Inv(\mathfrak{S}_n)=K(e_1,e_2,\dots,e_n).
	\]
	
	事实上,该引理是对称多项式基本定理\ref{thm:sym-basic}的一个推论.
\end{lemma}

有了该引理,我们就可构造满足$\Gal(f)=\mathfrak{S}_n$的多项式方程,事实上这就是一般$n$次方程.为了界定一般性,我们用不定元$t_1,t_2,\dots,t_n$来表示系数,则有以下定理.
\begin{lemma}\label{lem:trivial-equation}
	对于任一域$K$, $f(x)=x^n-t_1x^{n-1}+\dots+\left(-1\right)^nt_n$是域$K(t_1,t_2,\dots,t_n)$上的$n$次不可约可分多项式,则$\Gal(f)=\mathfrak{S}_n$.
\end{lemma}
\begin{proposition}
	$f$是$\mathbb{Q}$上的$p$次不可约多项式, $p$为素数,且$f$恰有两复根,则$\Gal(f)=\mathfrak{S}_p$.
\end{proposition}
\begin{example}[Brauer]
	设$p$是奇素数,$n_1<\cdots<n_{p-2}$均是偶数, $m$是满足$2m>\sum n_{i}^2$的正偶数.令
	\[
	f(x)=(x^2+m)(x-n_1)\cdots(x-n_{p-2})-2,
	\]
	则$\Gal(f,\mathbb{Q})=\mathfrak{S}_p$.
\end{example}
\begin{proposition}
	设域$K$的特征为零或与$n$互素,则$K$上多项式$f(x)=x^n-1$的Galois群$\Gal(f)\leqslant\mathbb{Z}_n^\times$,从而是Abel群.进一步,$\Gal(f)=\mathbb{Z}_n^\times$当且仅当分圆多项式$\Phi_n$是$K$上的不可约多项式.特别地,$\Gal(\Phi_n,\mathbb{Q})=\mathbb{Z}_n^\times$.
\end{proposition}

Galois反问题是代数和数论中富有意义和持久兴趣的研究课题,即:对于任意有限群$G$,找到域扩张使得其对应的Galois群恰为$G$.

\begin{corollary}
	利用Caylay定理\ref{thm:Caylay}和引理\ref{lem:sym-gal},任一有限群$G$均是某个域$K$上可分多项式的Galois群,其中$K$的特征任意.
\end{corollary}

然而,对于任意有限群$G$,是否存在Galois扩张$E/\mathbb{Q}$,使得$\Gal(E/\mathbb{Q})=G$?

1954年苏联数学家И. P. Шaфapeвич利用群论和代数数论证明了当$G$为有限可解群时答案是肯定的; J. G. Thompson证明了对于中心平凡且有有理强刚性共轭类组的有限群答案也是肯定的;他还证明了对魔群(monster group)的正确性.但对于任意有限群而言, Galois反问题依旧悬而未决.
\section{五次以上方程的根式不可解性}
\begin{definition}\index{yukuozhang@域扩张 (field extension)!根式扩张 (radical extension)}\index{ta@塔 (tower)}\index{ta@塔 (tower)!根塔 (radical tower)}
	域扩张$E/K$称为\textbf{根式扩张},若$E=K(d)$,并且存在自然数$n$使得$d^n\in F$.特别地,若$n=2$,则称为\textbf{二次根式扩张}.域扩张序列
	\[
	K_1\subseteq K_2\subseteq \cdots\subseteq K_r,
	\]
	称作域的扩张\textbf{塔},若满足$K_{i+1}/K_i(1\leqslant i\leqslant r-1)$是根式扩张,则称该塔为\textbf{根塔},若$K_{i+1}/K_i(1\leqslant i\leqslant r-1)$均为二次根式扩张,则称该根塔为\textbf{平方根塔}.
\end{definition}
\begin{definition}\index{genshikejie@根式可解 (radical sovable)}
	$f$是域$K$上的首一多项式, $E$为$f$在$K$上的分裂域,方程$f(x)=0$称作在$K$上\textbf{根式可解},是指存在根塔
	\[
	K=K_1\subseteq K_2\subseteq \cdots\subseteq K_r=F,
	\]
	且$E\subseteq F$.
\end{definition}
\begin{lemma}
	设$p$为素数,域$K$的特征不等于$p$, 若$p$次本原单位根$\zeta_p\in F$, $E/F$为$p$次循环Galois扩张,则$E/F$为根式扩张.
\end{lemma}
\begin{lemma}
	设域扩张$E/K$, $f$是$K$上的多项式,则$\Gal(f,E)\leqslant\Gal(f,K)$.
\end{lemma}
\begin{lemma}
	设$E/K$为有限可分扩张,$M$为$E$在$K$上的正规闭包.若$E/K$有根塔,则$M/K$也有根塔.
\end{lemma}

利用以上三个引理可以得到如下Galois大定理.
\begin{theorem}[Galois大定理]
	设$K$为特征零域, $f$是域$K$上的首一多项式,则方程$f(x)=0$在$K$上根式可解当且仅当$\Gal(f)$是可解群.
\end{theorem}

利用该定理和引理\ref{lem:trivial-equation}及推论\ref{prop:S_n-unsolvable}立刻得
\begin{theorem}[Abel-Ruffini]
	对于$n\geqslant5$, $K$为特征零域, $n$次一般方程$f(x)=x^n-t_1x^{n-1}+\dots+\left(-1\right)^nt_n=0$在$K(t_1,t_2,\dots,t_n)$上根式不可解.
\end{theorem}

	\appendix
	\chapter{法语语音初步}
\begin{quotation}
	\textit{君子知夫不全不粹之不足以为美也,故诵数以贯之,思索以通之,为其人以处之,除其害者以持养之.}
\end{quotation}

\section{法语与英语}
Évariste Galois是法国数学家,读音在法语中为/eva\textinvscr ist \textscriptg alwa/(IPA宽式音标).和通常的英语的发音区别很大,容易造成非法语学习者的困扰.

法语和英语虽都属于印欧语系,但英语属于日耳曼语族,法语属于罗曼语族.日耳曼语族还包括德语,荷兰语等.罗曼语族还包括拉丁语西班牙语,葡萄牙语,意大利语等.熟悉其中一两门语言的人可以窥见两种语族的明显区别.
\begin{figure}[htp]
	\centering
	\includegraphics[scale=0.27]{Origins}
	\caption{The percentage of modern English words derived from each language group}
\end{figure}

虽然英语在发展的过程中兼收并蓄了许多罗曼语族语言,但发音基本上遵循日耳曼语族的特点,仅在部分词汇中发源语言的读音.比如(此处音标为K.K.音标):
\begin{multicols}{3}
	\begin{itemize}
		\item avalanche /\textprimstress\ae v\textschwa l\textscripta\textlengthmark n\textesh/
		\item ballet /b\ae le\i\textlengthmark/
		\item bouquet /bu\textprimstress ke\i/
		\item bourgerois /ˈb\textupsilon ə\textyogh w\textscripta\textlengthmark/
		\item cliche /\textprimstress kli\textlengthmark\textesh e\i/
		\item façade /f\textschwa\textprimstress s\textscripta \textlengthmark d/
		\item genre /\textprimstress\textyogh\textscripta\textlengthmark r\textschwa/
		\item na\"ive /na\i \textprimstress i\textlengthmark v/
		\item rendezvous /\textprimstress r\textscripta\textlengthmark nd\i vu\textlengthmark/
		%\item souvenir /\textsecstress su\textlengthmark v\textschwa \textprimstress n\i r/
	\end{itemize}
\end{multicols}

还有一些具有明显法语特征的后缀,请注意这些单词的发音:
\begin{multicols}{3}
	\begin{enumerate}
		\setlength{\parsep}{0ex}
		\item eau,如:
		
		bureau(办公室)
		
		plateau(高原)
		
		tableau(场面)
		
		chapeau(帽子)
		
		beau(花花公子)
		
		nouveau(爆发户)
		
		\item ette,如:
		
		cigarette(烟卷)

		silhouette(剪影)
		
		croquette(油炸丸子)
		
		etiquette(礼仪)

		\item oir或oire, 如:
		
		memoir(回忆录)

		soiree(晚会)
		
		reservoir(水库)
		
		repertoire(全部节目)
		
		armoire(大橱)
		
		mouchoir(手帕)
		
		\item eur,如:
		
		amateur(业余爱好者)

		chauffeur(司机)
		
		grandeur(壮观)
		
		monsieur(先生)

		\item eon,如:
		
		dungeon(城堡)

		pigeon(鸽子)
		
		surgeon(外科医生)
		
		luncheon(午餐)
		
		\item et,如:
		
		ballet(芭蕾舞)

		beret(贝雷帽)
		
		buffet(小卖部)
		
		crochet(钩针编织品)
		
		bouquet(花束)
		
		croquet(棒球游戏)
		
		\item que,如:
		
		plaque(匾)

		clique(小集团)
		
		pique(生气)
		
		mosque(清真寺)
		
		unique(独一无二的)
		
		oblique(倾斜的)
		
		\item gue,如:
		
		fatigue(疲劳)

		vague(模糊的)
		
		vogue(时尚)
		
		plague(瘟疫)
		
		colleague(同事)
		
		league(联盟)
		
		\item ch,如:
		
		mustache(胡子)

		chef(男厨师长)
		
		brochure(小册子)
		
		parachute(降落伞)
		
		\item gn,如:
		
		assign(分配)

		campaign(战役)
		
		foreign(外国的)
		
		design(设计)
	\end{enumerate}
\end{multicols}
	
可见,这些单词的发音与通常的英语发音相差许多,在英语学习时也应多加注意.其余英语中的法语借词可参照:

\url{https://en.wikipedia.org/wiki/List_of_English_words_of_French_origin}

\section{法语的发音特点}

世界上的拼音文字可分为不需要音标拼写的直接拼法,需要音标辅助的间接拼法.世界上上绝大多数表音文字都是属于直接拼法,即是拼写都非常规则不使用音标就可以直接正确地拼读出单词,尽管法语的读音规则非常简单,但法语跟英语一样需要音标辅助拼写单词,法语属于间接拼法,当掌握规律后可以不用音标正确拼出单词,拼写比英语规则,通常在普通的法语字典里占一页的篇幅,但是法语单词中不发音的字母特别多,同一个字母或字母组合可以发不同的音,不同的字母或字母组合可以发相同的音,看单词一般可以读出正确的发音,但不一定能根据单词的发音正确拼写出单词,人们举例拼写复杂的言语时通常用法语和英语为例.

下面罗列一些法语的发音特点,以帮助汉语和英语学习者快速掌握法语词汇的发音.

\begin{itemize}
	\item 法语主要用五个变音符号,有时候用来表示不同的发音,有时候只是区别不同的语义:
	\subitem ``ˆ''长音符通常用于区分词形相同的词,或者表示某个元音字母后面曾经有一个被删去的字母,如êtes源于拉丁语单词estis(古法语为estes),中间的s已经随着语音流变而消失了;
	\subitem ``¨''分音符可以和多个元音字母组合,表示这个元音字母不跟前面的元音字母构成一个字母组合,而分别发音,类似于双元音;
	\subitem ``´''尖音符只用在字母``e''之上,表示这个字母发音为闭口音[e].也可以是某一个音消失的痕迹,如古法语系词的过去分词为esté(t),现代法语为été;
	\subitem ```''重音符用在字母``e''上表示这个字母发开口音[\textepsilon],而用在其他字母上则用以区分不同的语义,如ou(``或者'')和où(``哪里'')两个单词发音拼写完全一样,但是不同的词;
	\subitem ``¸''软音符只用于``c''字母下面,因为法语中和英语中一样,``c''在``a、o、u''前发[k]音,在``e、i''前发[s]音,如果在``a、o''想让它发[s]音,需加软音符,如在français(``法国人'')中.
	\item 单词末尾的辅音字母和e通常是不发音的,除非其后跟的有元音字母或同一个辅音字母.但是,这些辅音字母在联诵或者连音中可能发音.其次,当单词以字母f,l,r,c,q结尾时要发音,不过也有例外.最后,以双辅音如-gt,-ps,-ct等结尾,不过仍有例外.
	\item ``n''和``m''在元音字母前面发字母音,而在某些元音字母后面并且后面没有元音字母或者``m''或``n''相连的时候与前面的元音构成鼻化元音.
	\item 辅音字母``h''在任何时候都不发音,但在作为单词开头时区分为``哑音''和``嘘音'',词典上一般在嘘音单词前加上``*''.哑音和嘘音主要分别为哑音开头的词其读音和写法变化和元音开头的单词一样,而嘘音开头的单词的变化则和辅音开头的单词一样,即不能连读,不能省音等.
	\item 法语和英语、汉语的不同之处在于法语没有双元音,发每个元音时口型都不滑动,尤其要注意发鼻化元音时不能像汉语韵母似的有延续动作.法语的元音多数圆唇,因此法国人说话的时候嘴唇好像总是圆着的.
	\item p,t,k分别发/p,t,k/,即不送气音.注意汉语普通话中p,t,k为送气音,而b,d,g为不送气音,这些都为清音,汉语普通话中不存在浊音,而送气音是法语中没有的.英语在流变中也逐渐失去了浊音,大多数浊音用不送气清音/p,t,k/代替,而原本的p,t,k与汉语普通话一样发送气清音, IPA严式音标记作/p$^\text{h}$,t$^\text{h}$,k$^\text{h}$/.清音和浊音的区别是声带是否振动,但对汉语言学习者来说浊音发音较为困难,在不引起混淆的情况下可以用不送气清音代替.
	\item 法语中辅音j发/\textyogh/,即浊腭龈擦音,类似于汉语拼音中r的音.英语中此音也有被拼写为ge,多是来自法语的外来词,例如genre,garage,prestige以及Baton Rouge.英语、德语中,此音通常被拼写为zh,但主要用于外来词.例如,Zhukovsky(茹科夫斯基)、Brezhnev(勃列日涅夫)和Zhengzhou(郑州).zhoosh可能是英语中唯一的含有此音的本族词.
	\item 前词是 ce/de/je/jusque/le/la/me/ne/que/se/te,后词是元音或者哑音h开头时会出现省音,例:ce est=c'est, de aimer=d'aimer, je aime=j'aime,此外la, si, jusque, lorsque, presque, puisque等词也有省音现象.
\end{itemize}

其余具体发音规则可参照法语正字法的维基百科界面:

\url{https://en.wikipedia.org/wiki/French_orthography}

\section{法国数学家人名例}

有了以上的理论知识已经足够回答最初的问题,Galois中的Gal与英语发音相似,而oi发/w\textscripta/的音,末尾辅音s不发音,这也正应和了汉语音译伽(\textscriptg \=\textscripta)罗瓦.以下再列举一些常见的法国数学家及其译名,结合上面的规则,感受它们的发音:

\begin{multicols}{3}
	\begin{itemize}
		\item Baire 贝尔
		
		(``拜尔''为错译)
		\item Bézout 贝祖
		\item Binet 比内
		\item Bourbaki 布尔巴基
		\item Cartan 嘉当
		\item Cauchy 柯西
		\item d'Alembert 达朗贝尔
		\item Darboux 达布
		\item de Moivre 棣莫佛
		\item Descartes 笛卡尔
		\item Fatou 法图
		\item Fermat 费马
		\item Fréchet 弗雷歇
		\item Goursat 古尔萨
		\item Grothendieck 
		
		格罗滕迪克
		\item Hadamard 阿达马
		\item Hermite 厄米特
		\item Jordan 若尔当
		\item Lagrange 拉格朗日
		\item Laplace 拉普拉斯
		\item Laurent 洛朗
		\item Lebesgue 勒贝格
		\item Legendre 勒让德
		\item l'H\^opital 洛必达
		\item Liouville 刘维尔
		\item Mandelbrot 曼德博
		\item Monge 蒙日
		\item Parseval 帕塞瓦尔
		\item Pascal 帕斯卡
		\item Picard 皮卡
		\item Poincaré 庞加莱
		\item Poisson 泊松
		\item Rolle 罗尔
		\item Rouché 鲁歇
		\item Serre 塞尔
		\item Sturm 斯图姆
		\item Vandermonde 范德蒙
		\item Viète 韦达
		\item Weil 韦伊
		\item Wroński 朗斯基
	\end{itemize}
\end{multicols}
可见汉语音译始终遵循``名从主人''的原则,这不光对法语人名来说是这样,对于其他语言同样如此,需要认真区分.

在常见的数学家中,来自非英语国家诸如古希腊,德国,苏俄,印度,日本的人名或词汇要多加注意,尽量遵循源语言的读法.


	\chapter{拉丁转写对照}
本附录参考维基百科罗列了希腊语,俄语和日语的拉丁转写,并以各国数学家人名为例,具体参照

\url{https://mathpron.github.io}

本附录采用宽式IPA,有关国际音标(IPA)的内容请参考\cite{LT}以及

\url{https://en.wikipedia.org/wiki/International_Phonetic_Alphabet}

\url{https://www.bilibili.com/video/BV1QA411i7Yf}

\includepdf[pagecommand={\thispagestyle{plain}},pages=-]{lin}
%addtotoc={{1,2},{希腊语},1,{希腊语},{Greek}}
	\backmatter
	{%\footnotesize
	\indexprologue{兹给出名词索引及其英文翻译,以供参考.中文术语按汉语拼音排序.}
	\printindex}
	\bibliography{Galois}
\end{document}