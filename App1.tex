\chapter{法语语音初步}
\begin{quotation}
	\textit{君子知夫不全不粹之不足以为美也,故诵数以贯之,思索以通之,为其人以处之,除其害者以持养之.}
\end{quotation}

\section{法语与英语}
Évariste Galois是法国数学家,读音在法语中为/eva\textinvscr ist \textscriptg alwa/(IPA宽式音标).和通常的英语的发音区别很大,容易造成非法语学习者的困扰.

法语和英语虽都属于印欧语系,但英语属于日耳曼语族,法语属于罗曼语族.日耳曼语族还包括德语,荷兰语等.罗曼语族还包括拉丁语西班牙语,葡萄牙语,意大利语等.熟悉其中一两门语言的人可以窥见两种语族的明显区别.
\begin{figure}[htp]
	\centering
	\includegraphics[scale=0.27]{Origins}
	\caption{The percentage of modern English words derived from each language group}
\end{figure}

虽然英语在发展的过程中兼收并蓄了许多罗曼语族语言,但发音基本上遵循日耳曼语族的特点,仅在部分词汇中发源语言的读音.比如(此处音标为K.K.音标):
\begin{multicols}{3}
	\begin{itemize}
		\item avalanche /\textprimstress\ae v\textschwa l\textscripta\textlengthmark n\textesh/
		\item ballet /b\ae le\i\textlengthmark/
		\item bouquet /bu\textprimstress ke\i/
		\item bourgerois /ˈb\textupsilon ə\textyogh w\textscripta\textlengthmark/
		\item cliche /\textprimstress kli\textlengthmark\textesh e\i/
		\item façade /f\textschwa\textprimstress s\textscripta \textlengthmark d/
		\item genre /\textprimstress\textyogh\textscripta\textlengthmark r\textschwa/
		\item na\"ive /na\i \textprimstress i\textlengthmark v/
		\item rendezvous /\textprimstress r\textscripta\textlengthmark nd\i vu\textlengthmark/
		%\item souvenir /\textsecstress su\textlengthmark v\textschwa \textprimstress n\i r/
	\end{itemize}
\end{multicols}

还有一些具有明显法语特征的后缀,请注意这些单词的发音:
\begin{multicols}{3}
	\begin{enumerate}
		\setlength{\parsep}{0ex}
		\item eau,如:
		
		bureau(办公室)
		
		plateau(高原)
		
		tableau(场面)
		
		chapeau(帽子)
		
		beau(花花公子)
		
		nouveau(爆发户)
		
		\item ette,如:
		
		cigarette(烟卷)

		silhouette(剪影)
		
		croquette(油炸丸子)
		
		etiquette(礼仪)

		\item oir或oire, 如:
		
		memoir(回忆录)

		soiree(晚会)
		
		reservoir(水库)
		
		repertoire(全部节目)
		
		armoire(大橱)
		
		mouchoir(手帕)
		
		\item eur,如:
		
		amateur(业余爱好者)

		chauffeur(司机)
		
		grandeur(壮观)
		
		monsieur(先生)

		\item eon,如:
		
		dungeon(城堡)

		pigeon(鸽子)
		
		surgeon(外科医生)
		
		luncheon(午餐)
		
		\item et,如:
		
		ballet(芭蕾舞)

		beret(贝雷帽)
		
		buffet(小卖部)
		
		crochet(钩针编织品)
		
		bouquet(花束)
		
		croquet(棒球游戏)
		
		\item que,如:
		
		plaque(匾)

		clique(小集团)
		
		pique(生气)
		
		mosque(清真寺)
		
		unique(独一无二的)
		
		oblique(倾斜的)
		
		\item gue,如:
		
		fatigue(疲劳)

		vague(模糊的)
		
		vogue(时尚)
		
		plague(瘟疫)
		
		colleague(同事)
		
		league(联盟)
		
		\item ch,如:
		
		mustache(胡子)

		chef(男厨师长)
		
		brochure(小册子)
		
		parachute(降落伞)
		
		\item gn,如:
		
		assign(分配)

		campaign(战役)
		
		foreign(外国的)
		
		design(设计)
	\end{enumerate}
\end{multicols}
	
可见,这些单词的发音与通常的英语发音相差许多,在英语学习时也应多加注意.其余英语中的法语借词可参照:

\url{https://en.wikipedia.org/wiki/List_of_English_words_of_French_origin}

\section{法语的发音特点}

世界上的拼音文字可分为不需要音标拼写的直接拼法,需要音标辅助的间接拼法.世界上上绝大多数表音文字都是属于直接拼法,即是拼写都非常规则不使用音标就可以直接正确地拼读出单词,尽管法语的读音规则非常简单,但法语跟英语一样需要音标辅助拼写单词,法语属于间接拼法,当掌握规律后可以不用音标正确拼出单词,拼写比英语规则,通常在普通的法语字典里占一页的篇幅,但是法语单词中不发音的字母特别多,同一个字母或字母组合可以发不同的音,不同的字母或字母组合可以发相同的音,看单词一般可以读出正确的发音,但不一定能根据单词的发音正确拼写出单词,人们举例拼写复杂的言语时通常用法语和英语为例.

下面罗列一些法语的发音特点,以帮助汉语和英语学习者快速掌握法语词汇的发音.

\begin{itemize}
	\item 法语主要用五个变音符号,有时候用来表示不同的发音,有时候只是区别不同的语义:
	\subitem ``ˆ''长音符通常用于区分词形相同的词,或者表示某个元音字母后面曾经有一个被删去的字母,如êtes源于拉丁语单词estis(古法语为estes),中间的s已经随着语音流变而消失了;
	\subitem ``¨''分音符可以和多个元音字母组合,表示这个元音字母不跟前面的元音字母构成一个字母组合,而分别发音,类似于双元音;
	\subitem ``´''尖音符只用在字母``e''之上,表示这个字母发音为闭口音[e].也可以是某一个音消失的痕迹,如古法语系词的过去分词为esté(t),现代法语为été;
	\subitem ```''重音符用在字母``e''上表示这个字母发开口音[\textepsilon],而用在其他字母上则用以区分不同的语义,如ou(``或者'')和où(``哪里'')两个单词发音拼写完全一样,但是不同的词;
	\subitem ``¸''软音符只用于``c''字母下面,因为法语中和英语中一样,``c''在``a、o、u''前发[k]音,在``e、i''前发[s]音,如果在``a、o''想让它发[s]音,需加软音符,如在français(``法国人'')中.
	\item 单词末尾的辅音字母和e通常是不发音的,除非其后跟的有元音字母或同一个辅音字母.但是,这些辅音字母在联诵或者连音中可能发音.其次,当单词以字母f,l,r,c,q结尾时要发音,不过也有例外.最后,以双辅音如-gt,-ps,-ct等结尾,不过仍有例外.
	\item ``n''和``m''在元音字母前面发字母音,而在某些元音字母后面并且后面没有元音字母或者``m''或``n''相连的时候与前面的元音构成鼻化元音.
	\item 辅音字母``h''在任何时候都不发音,但在作为单词开头时区分为``哑音''和``嘘音'',词典上一般在嘘音单词前加上``*''.哑音和嘘音主要分别为哑音开头的词其读音和写法变化和元音开头的单词一样,而嘘音开头的单词的变化则和辅音开头的单词一样,即不能连读,不能省音等.
	\item 法语和英语、汉语的不同之处在于法语没有双元音,发每个元音时口型都不滑动,尤其要注意发鼻化元音时不能像汉语韵母似的有延续动作.法语的元音多数圆唇,因此法国人说话的时候嘴唇好像总是圆着的.
	\item p,t,k分别发/p,t,k/,即不送气音.注意汉语普通话中p,t,k为送气音,而b,d,g为不送气音,这些都为清音,汉语普通话中不存在浊音,而送气音是法语中没有的.英语在流变中也逐渐失去了浊音,大多数浊音用不送气清音/p,t,k/代替,而原本的p,t,k与汉语普通话一样发送气清音, IPA严式音标记作/p$^\text{h}$,t$^\text{h}$,k$^\text{h}$/.清音和浊音的区别是声带是否振动,但对汉语言学习者来说浊音发音较为困难,在不引起混淆的情况下可以用不送气清音代替.
	\item 法语中辅音j发/\textyogh/,即浊腭龈擦音,类似于汉语拼音中r的音.英语中此音也有被拼写为ge,多是来自法语的外来词,例如genre,garage,prestige以及Baton Rouge.英语、德语中,此音通常被拼写为zh,但主要用于外来词.例如,Zhukovsky(茹科夫斯基)、Brezhnev(勃列日涅夫)和Zhengzhou(郑州).zhoosh可能是英语中唯一的含有此音的本族词.
	\item 前词是 ce/de/je/jusque/le/la/me/ne/que/se/te,后词是元音或者哑音h开头时会出现省音,例:ce est=c'est, de aimer=d'aimer, je aime=j'aime,此外la, si, jusque, lorsque, presque, puisque等词也有省音现象.
\end{itemize}

其余具体发音规则可参照法语正字法的维基百科界面:

\url{https://en.wikipedia.org/wiki/French_orthography}

\section{法国数学家人名例}

有了以上的理论知识已经足够回答最初的问题,Galois中的Gal与英语发音相似,而oi发/w\textscripta/的音,末尾辅音s不发音,这也正应和了汉语音译伽(\textscriptg \=\textscripta)罗瓦.以下再列举一些常见的法国数学家及其译名,结合上面的规则,感受它们的发音:

\begin{multicols}{3}
	\begin{itemize}
		\item Baire 贝尔
		
		(``拜尔''为错译)
		\item Bézout 贝祖
		\item Binet 比内
		\item Bourbaki 布尔巴基
		\item Cartan 嘉当
		\item Cauchy 柯西
		\item d'Alembert 达朗贝尔
		\item Darboux 达布
		\item de Moivre 棣莫佛
		\item Descartes 笛卡尔
		\item Fatou 法图
		\item Fermat 费马
		\item Fréchet 弗雷歇
		\item Goursat 古尔萨
		\item Grothendieck 
		
		格罗滕迪克
		\item Hadamard 阿达马
		\item Hermite 厄米特
		\item Jordan 若尔当
		\item Lagrange 拉格朗日
		\item Laplace 拉普拉斯
		\item Laurent 洛朗
		\item Lebesgue 勒贝格
		\item Legendre 勒让德
		\item l'H\^opital 洛必达
		\item Liouville 刘维尔
		\item Mandelbrot 曼德博
		\item Monge 蒙日
		\item Parseval 帕塞瓦尔
		\item Pascal 帕斯卡
		\item Picard 皮卡
		\item Poincaré 庞加莱
		\item Poisson 泊松
		\item Rolle 罗尔
		\item Rouché 鲁歇
		\item Serre 塞尔
		\item Sturm 斯图姆
		\item Vandermonde 范德蒙
		\item Viète 韦达
		\item Weil 韦伊
		\item Wroński 朗斯基
	\end{itemize}
\end{multicols}
可见汉语音译始终遵循``名从主人''的原则,这不光对法语人名来说是这样,对于其他语言同样如此,需要认真区分.

在常见的数学家中,来自非英语国家诸如古希腊,德国,苏俄,印度,日本的人名或词汇要多加注意,尽量遵循源语言的读法.

